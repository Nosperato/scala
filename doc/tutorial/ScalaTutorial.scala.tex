%% This file really contains -*- LaTeX -*- code, to be processed by
%% the scalatex script.

%% $Id$

\documentclass[a4paper,12pt]{article}

\usepackage{palatino}
\usepackage{alltt}
\usepackage{xspace}
\usepackage{url}

\newcommand{\langname}[1]{#1\xspace}

\newcommand{\Scala}{\langname{Scala}}
\newcommand{\Java}{\langname{Java}}

\newcommand{\toolname}[1]{\texttt{#1}\xspace}

\newcommand{\scalac}{\toolname{scalac}}
\newcommand{\java}{\toolname{java}}

\newcommand{\ident}[1]{\url{#1}\xspace}

\begin{document}

\title{An introduction to \Scala\\[.5em]\normalsize(for \Java programmers)}
\author{Michel Schinz}
\maketitle

\section{Introduction}
\label{sec:introduction}

This document gives a quick introduction to the \Scala language and
compiler. It is intended for people who have already some experience
at programming and want an overview of what they can do with \Scala. A
basic knowledge of object-oriented programming, especially in \Java,
is assumed.

\section{A first example}
\label{sec:first-example}

As a first example, we will use the standard \emph{Hello world}
program, which is not very fascinating but makes it easy to
demonstrate the use of the \Scala tools without knowing too much about
the language. Here is how it looks:
\begin{scalaprogram}{HelloWorld}
object HelloWorld {
  def main(args: Array[String]): unit = {
    System.out.println("Hello, world!");
  }
}
\end{scalaprogram}

The structure of this program should be familiar to Java programmers:
it consists of one method called \ident{main} which takes the command
line arguments, an array of strings, as parameter; the body of this
method consists of a single call to the \ident{println} method of the
object representing the standard output, with the friendly greeting as
argument. The \ident{main} method is declared as returning a value of
type \ident{unit}, which for now can be seen as similar to \Java's
\ident{void} type.

What should be less familiar to Java programmers is the \ident{object}
declaration containing the \ident{main} method. Such a declaration
introduces what is commonly known as a \emph{singleton object}, that
is a class with a single instance. The declaration above thus declares
both a class called \ident{HelloWorld} and an instance of that class,
also called \ident{HelloWorld}. This instance is created lazily, the
first time it is used.

The astute reader might also have noticed that the \ident{main} method
is not declared as \ident{static} here. This is because static members
(methods or fields) do not exist in \Scala. Rather than define static
members, the \Scala programmer declares these members in singleton
objects.

\subsection{Compiling the example}
\label{sec:compiling-example}

To compile the example, we need to use \scalac, the \Scala compiler.
\scalac works like most compilers: it takes a source file as argument,
maybe some options, and produces one or several object files. The
object files it produces are standard \Java class files.

If we save the above program in a file called
\ident{HelloWorld.scala}, we can compile it by issuing the following
command (the greater-than sign `\verb|>|' represents the shell prompt
and should not be typed):
\begin{verbatim}
> scalac HelloWorld.scala
\end{verbatim}
This will generate a few class files in the current directory, one of
which called \ident{HelloWorld.class}. This file contains a class
which can be directly executed using the \java command, as will be
seen in the following section.

\subsection{Running the example}
\label{sec:running-example}

Once compiled, a \Scala program can be run like a \Java program, using
the \java command. However, a compiled \Scala program needs to access
some support classes at run-time, which should be available through
\Java's class path. These support classes are distributed in a JAR
file called \url{scala.jar}, which lives in the directory
\url{SCALA_HOME/lib}. Here \url{SCALA_HOME} is a place-holder for the
name of the directory where the \Scala distribution was installed.

The example above can therefore be executed using the command below,
if we assume that the \Scala distribution was installed in
\url{/usr/local}:
\begin{verbatim}
> java -classpath /usr/local/lib/scala.jar:. HelloWorld
\end{verbatim}
\scalaprogramoutput{HelloWorld}

\section{Interaction with Java}
\label{sec:inter-with-java}

One of the strength of \Scala is that it makes it very easy to
interact with \Java code. Actually, the example of the previous
section showed this: to print the message on screen, we simply used a
call to \Java's \ident{println} method on the (\Java) object
\ident{System.out}.

All \Java code is accessible as easily from \Scala. Of course, it is
sometimes necessary to import classes, as one does in \Java, in order
to be able to use them. All classes in the \ident{java.lang} packages
are imported by default, others need to be imported explicitly.

Let's look at another example to see this. The aim of this example is
to compute and print the factorial of 100 using \Java big integers
(i.e. the class \ident{java.math.BigInteger}), since the result does
not fit in a \Java integer. This program looks like this:
\begin{scalaprogram}{BigFactorial}
object BigFactorial {
  import java.math.BigInteger, BigInteger._;

  def fact(x: BigInteger): BigInteger =
    if (x == ZERO) ONE
    else x multiply fact(x subtract ONE);

  def main(args: Array[String]): unit =
    System.out.println("fact(100) = "
                       + fact(new BigInteger("100")));
}
\end{scalaprogram}

\Scala's \ident{import} statement looks very similar to \Java's
equivalent, but an important difference appears here: to import all
the names of a package or class, one uses the underscore (\verb|_|)
character instead of the asterisk (\verb|*|). This is due to the fact
that, as we will see later, the asterisk is actually a valid \Scala
identifier.

The \ident{import} statement above therefore starts by importing the
class \ident{java.math.BigInteger}, and then all the names it
contains. This makes the static fields \ident{ZERO} and \ident{ONE}
directly visible.

The \ident{fact} method also shows some characteristics of \Scala's
syntax. The first one is that the method body does not have to be
surrounded by curly braces if it consists of a single expression.
The second one is that methods taking one argument can be used with an
infix syntax. That is, the expression
\begin{verbatim}
x subtract ONE
\end{verbatim}
is just another, slightly less verbose way of writing the expression
\begin{verbatim}
x.subtract(ONE)
\end{verbatim}
This might seem like a minor syntactic detail, but it has important
consequences, one of which will be explored in the next section.

To conclude this section about integration with \Java, it should be
noted that it is also possible to inherit from \Java classes and
implement \Java interfaces directly in \Scala.

\section{Everything is an object}
\label{sec:everything-an-object}

\Scala is a pure object-oriented language in the sense that
\emph{everything} is an object, including numbers or functions. It
differs from \Java in that respect, since \Java distinguishes numeric
types from objects, and does not enable one to manipulate functions as
values.

\subsection{Numbers are objects}
\label{sec:numbers-are-objects}

Since numbers are objects, they also have methods. And in fact, an
arithmetic expression like the following:
\begin{verbatim}
1 + 2 * 3 / x
\end{verbatim}
consists exclusively of method calls, because it is equivalent to the
following expression, as we saw in the previous section:
\begin{verbatim}
1.+(2.*(3./(x)))
\end{verbatim}
This also means that \ident{+}, \ident{*}, etc. are legal identifiers
in \Scala.

\subsection{Functions are objects}
\label{sec:funct-are-objects}

Perhaps more surprising for the \Java programmer, functions are also
objects in \Scala. It is therefore possible to pass functions as
arguments, to store them in variables, and to return them from other
functions. This ability to manipulate functions as values is one of
the cornerstone of a very interesting programming paradigm called
\emph{functional programming}.

As a very simple example of why it can be useful to use functions as
values, let's consider a timer function whose aim is to perform some
action every second. How do we pass it the action to perform? Quite
logically, as a function. This very simple kind of function passing
should be familiar to many programmers: it is often used in
user-interface code, to register call-back functions which get called
when some event occurs.

In the following program, the timer function is called
\ident{oncePerSecond}, and it gets a call-back function as argument.
The type of this function is written \verb|() => unit| and is the type
of all functions which have no arguments and return a value of type
\ident{unit}. The main function of this program simply calls this
timer function with a call-back which prints a sentence on the
terminal. In other words, this program endlessly prints the sentence
\emph{time flies like an arrow} every second.

\begin{scalaprogram}{Timer}
object Timer {
  def oncePerSecond(callback: () => unit): unit =
    while (true) { callback(); Thread sleep 1000 };

  def timeFlies(): unit =
    System.out.println("time flies like an arrow...");

  def main(args: Array[String]): unit =
    oncePerSecond(timeFlies);
}
\end{scalaprogram}

\subsubsection{Anonymous functions}
\label{sec:anonymous-functions}

While this program is easy to understand, it can be refined a bit.
First of all, notice that the function \ident{timeFlies} is only
defined in order to be passed later to the \ident{oncePerSecond}
function. Having to give a name to that function, which is only used
once, might seem unnecessary, and it would in fact be nice to be able
to construct this function just as it is passed to
\ident{oncePerSecond}. This is possible in \Scala using
\emph{anonymous functions}, which are exactly that: functions without
a name. The revised version of our timer program using an anonymous
function instead of \ident{timeFlies} looks like that:
\begin{scalaprogram}{TimerAnonymous}
object TimerAnonymous {
  def oncePerSecond(callback: () => unit): unit =
    while (true) { callback(); Thread sleep 1000 };

  def main(args: Array[String]): unit =
    oncePerSecond(() =>
      System.out.println("time flies like an arrow..."));
}
\end{scalaprogram}
The presence of an anonymous function in this example is revealed by
the right arrow `\verb|=>|' which separates the function's argument
list from its body. In this example, the argument list is empty, as
witnessed by the empty pair of parenthesis on the left of the arrow.
The body of the function is the same as the one of \ident{timeFlies}
above.

% TODO fonctions avec environnement

\section{Classes}
\label{sec:classes}

As we have seen above, \Scala is an object-oriented language, and as
such it has a concept of class.\footnote{For the sake of completeness,
  it should be noted that some object-oriented languages do not have
  the concept of class, but \Scala is not one of them.}
Classes in \Scala are declared using a syntax which is close to
\Java's syntax. One important difference is that classes in \Scala can
have parameters. This is illustrated in the following definition of
complex numbers.
\begin{scalaprogram}{Complex}
class Complex(real: double, imaginary: double) {
  def re() = real;
  def im() = imaginary;
}
\end{scalaprogram}
This complex class takes two arguments, which are the real and
imaginary part of the complex. It then defines two methods, called
\ident{re} and \ident{im} which give access to these two parts.

It should be noted that the return type of these two methods is not
given explicitly. It will be inferred automatically by the compiler,
which looks at the right-hand side of these methods and deduces that
both return a value of type \ident{double}.

The compiler is not always able to infer types like it does here, and
there is unfortunately no simple rule to know exactly when it will be,
and when not. In practice, this is usually not a problem since the
compiler complains when it is not able to infer a type which was not
given explicitly. As a simple rule, beginner \Scala programmers
should try to omit type declarations which seem superfluous to them,
because they are easily deduced from the context, and see whether the
compiler agrees with them. After some time, they should get a good
feeling about when to omit types, and when to specify them
explicitly.

\subsection{Methods without arguments}
\label{sec:meth-wo-args}

A small problem of the methods \ident{re} and \ident{im} is that, in
order to call them, one has to put an empty pair of parenthesis after
their name, as the following example shows:
\begin{alltt}
val c = new Complex(1.2, 3.4);
System.out.println("imaginary part: " + \underline{c.im()});
\end{alltt}
It would be nicer to be able to access the real and imaginary parts
like if they were fields, without putting the empty pair of
parenthesis. This is perfectly doable in \Scala, simply by defining
the methods as methods \emph{without arguments}. These differ from
methods with zero arguments in that they don't have parenthesis after
their name, neither in their definition nor in their use. Our
\ident{Complex} class can be rewritten as follows:
\begin{scalaprogram}{Complex2}
class Complex(real: double, imaginary: double) {
  def re = real;
  def im = imaginary;
}
\end{scalaprogram}

\subsection{Inheritance and overriding}
\label{sec:inheritance}

All classes in \Scala inherit from a super-class. When no super-class
is specified, as in the \ident{Complex} example of previous section,
\ident{scala.Object} is implicitly used.

It is possible to override methods inherited from a super-class in
\Scala. It is however mandatory to explicitly specify that a method
overrides another one using the \ident{override} modifier, in order to
avoid accidental overriding. As an example, our \ident{Complex} class
can be augmented with a redefinition of the \ident{toString} method
inherited from \ident{Object}.
\begin{scalaprogram}{Complex3}
class Complex(real: double, imaginary: double) {
  def re = real;
  def im = imaginary;
  override def toString() =
    "" + re + (if (im < 0) "-" else "+") + im + "i";
}
\end{scalaprogram}

\section{Mixins}
\label{sec:mixins}

Apart from inheriting code from a super-class, a \Scala class can also
import code from one or several \emph{mixins}.

\section{Case classes and pattern matching}
\label{sec:case-classes-pattern}


\section{Genericity}
\label{sec:genericity}



\section{Conclusion}
\label{sec:conclusion}

This document gave a quick overview of the \Scala language and
presented some basic examples. The interested reader can go on by
reading the companion document \textit{Scala By Example\/} and consult
the \textit{Scala Language Specification\/} when needed.

\end{document}
