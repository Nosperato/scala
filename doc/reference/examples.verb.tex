%% $Id$

\documentclass[11pt]{report}

\usepackage{fleqn,a4wide,vquote,modefs,math,prooftree,scaladefs}
\newcommand{\exercise}{\paragraph{Exercise:}}

\title{Scala By Examples}

\author{
Martin Odersky
}

\sloppy
\begin{document}
\maketitle

\chapter{\label{sec:example-one}A First Example}

As a first example, here is an implementation of Quicksort in Scala.
\begin{verbatim}
def sort(xs: Array[Int]): Unit = {

  def swap(i: Int, j: Int): Unit = {
    val t = xs(i); xs(i) = xs(j); xs(j) = t;
  }

  def sort1(l: Int, r: Int): Unit = {
    val pivot = xs((l + r) / 2);
    var i = l, j = r;
    while (i <= j) {
      while (xs(i) < pivot) { i = i + 1 }
      while (xs(j) > pivot) { j = j - 1 }
      if (i <= j) { 
        swap(i, j);
        i = i + 1;
        j = j - 1;
      }
    } 
    if (l < j) sort1(l, j);
    if (j < r) sort1(i, r);
  }

  sort1(0, xs.length - 1);
}
\end{verbatim}
The implementation looks quite similar to what one would write in Java
or C.  We use the same operators and similar control structures.
There are also some minor syntactical differences. In particular:
\begin{itemize}
\item
Definitions start with a reserved word. Function definitions start
with \verb@def@, variable definitions start with \verb@var@ and
definitions of values (i.e. read only variables) start with \verb@val@.
\item
The declared type of a symbol is given after the symbol and a colon.
The declared type can often be omitted, because the compiler can infer
it from the context.
\item
We use \verb@Unit@ instead of \verb@void@ to define the result type of
a procedure.
\item
Array types are written \verb@Array[$T$]@ rather than \verb@$T$[]@.
\item
Functions can be nested inside other functions. Nested functions can
access parameters and local variables of enclosing functions. For
instance, the name of the array \verb@a@ is visible in functions
\verb@swap@ and \verb@sort1@, and therefore need not be passed as a
parameter to them.
\end{itemize}
So far, Scala looks like a fairly conventional language with some
syntactic particularitis. In fact it is possible to write programs in a
conventional imperative or object-oriented style. This is important
because it is one of the things that makes it easy to combine Scala
components with components written in mainstream languages such as
Java, C\# or Visual Basic.

However, it is also possible to write programs in a style which looks
completely different. Here is Quicksort again, this time written in
functional style.

\begin{verbatim}
def sort(xs: List[Int]): List[Int] = {
  val pivot = a(a.length / 2);
  sort(a.filter(x => x < pivot))
    :::  a.filter(x => x == pivot)
    :::  sort(a.filter(x => x > pivot))
}
\end{verbatim}

The functional program works with lists instead of arrays\footnote{
This is necessary for now, because arrays do not yet support
\verb@filter@ and \verb@:::@.}.
It captures the essence of the quicksort algorithm in a concise way:
\begin{itemize}
\item Pick an element in the middle of the list as a pivot.
\item Partition the lists into two sub-lists containing elements that
are less than, respectively greater than the pivot element, and a
third list which contains elements equal to privot.
\item Sort the first two sub-lists by a recursive invocation of
the sort function\footnote{This is not quite what the imperative algorithm does;
the latter partitions the array into two sub-arrays containing elements
less than or greater or equal to pivot.}
\item The result is obtained by appending the three sub-lists together.
\end{itemize}
Both the imperative and the functional implementation have the same
asymptotic complexity -- $O(N;log(N))$ in the average case and
$O(N^2)$ in the worst case. But where the imperative implementation
operates in place by modifying the argument array, the functional
implementation returns a new sorted list and leaves the argument
list unchanged. The functional implementation thus requires more
transient memory than the imperative one.

The functional implementation makes it look like Scala is a language
that's specialized for functional operations on lists. In fact, it
is not; all of the operations used in the example are simple library
methods of a class \verb@List[t]@ which is part of the standard
Scala library, and which itself is implemented in Scala.

In particular, there is the method \verb@filter@ which takes as
argument a {\em predicate function} that maps list elements to
Boolean values. The result of \verb@filter@ is a list consisting of
all the elements of the original list for which the given predicate
function is true.  The \verb@filter@ method of an object of type
\verb@List[t]@ thus has the signature
\begin{verbatim}
  def filter(p: t => Boolean): List[t]  .
\end{verbatim}
Here, \verb@t => Boolean@ is the type of functions that take an element
of type \verb@t@ and return a \verb@Boolean@.  Functions like
\verb@filter@ that take another function as argument or return one as
result are called {\em higher-order} functions.

In the quicksort program, \verb@filter@ is applied three times to an
anonymous function argument.  The first argument,
\verb@x => x <= pivot@ represents the function that maps its parameter
\verb@x@ to the Boolean value \verb@x <= pivot@. That is, it yields
true if \verb@x@ is smaller or equal than \verb@pivot@, false
otherwise. The function is anonymous, i.e.\ it is not defined with a
name. The type of the \verb@x@ parameter is omitted because a Scala
compiler can infer it automatically from the context where the
function is used. To summarize, \verb@xs.filter(x => x <= pivot)@
returns a list consisting of all elements of the list \verb@xs@ that are
smaller than \verb@pivot@.

It is also possible to apply higher-order functions such as
\verb@filter@ to named function arguments. Here is functional
quicksort again, where the two anonymous functions are replaced by
named auxiliary functions that compare the argument to the
\verb@pivot@ value.

\begin{verbatim}
def sort (xs: List[Int]): List[Int] = {
  val pivot = xs(xs.length / 2);
  def leqPivot(x: Int) = x <= pivot;
  def gtPivot(x: Int) = x > pivot;
  def eqPivot(x: Int) = x == pivot;
  sort(xs filter leqPivot)  
    ::: sort(xs filter eqPivot)  
    ::: sort(xs filter gtPivot)
}
\end{verbatim}

An object of type \verb@List[t]@ also has a method ``\verb@:::@''
which takes an another list and which returns the result of appending this
list to itself. This method has the signature
\begin{verbatim}
  def :::(that: List[t]): List[t] .
\end{verbatim}
Scala does not distinguish between identifiers and operator names. An
identifier can be either a sequence of letters and digits which begins
with a letter, or it can be a sequence of special characters, such as
``\verb@+@'', ``\verb@*@'', or ``\verb@:@''.  The last definition thus
introduced a new method identifier ``\verb@:::@''.  This identifier is
used in the Quicksort example as a binary infix operator that connects
the two sub-lists resulting from the partition. In fact, any method
can be used as an operator in Scala.  The binary operation $E;op;E'$
is always interpreted as the method call $E.op(E')$. This holds also
for binary infix operators which start with a letter. The recursive call
to \verb@sort@ in the last quicksort example is hence equivalent to:
\begin{verbatim}
  sort(a.filter(leqPivot))
    .append(sort(a.filter(eqPivot)))
    .append(sort(a.filter(gtPivot))) .
\end{verbatim}

Looking again in detail at the first, imperative implementation of
Quicksort, we find that many of the language constructs used in the
second solution are also present, albeit in a disguised form.

For instance, ``standard'' binary operators such as \verb@+@,
\verb@-@, or \verb@<@ are not treated in any special way. Like
\verb@append@, they are methods of their left operand. Consequently,
the expression \verb@i + 1@ is regarded as the invocation
\verb@i.+(1)@ of the \verb@+@ method of the integer value \verb@x@.
Of course, a compiler is free (if it is moderately smart, even expected)
to recognize the special case of calling the \verb@+@ method over
integer arguments and to generate efficient inline code for it.

Control constructs such as \verb@while@ are also not primitive but are
predefined functions in the standard Scala library. Here is the
definition of \verb@while@ in Scala.
\begin{verbatim}
def while (def p: Boolean) (def s: Unit): Unit = if (p) { s ; while(p)(s) }
\end{verbatim}
The \verb@while@ function takes as first parameter a test function,
which takes no parameters and yields a Boolean value. As second
parameter it takes a command function which also takes no parameters
and yields a trivial result. \verb@while@ invokes the command function
as long as the test function yields true. Again, compilers are free to
pick specialized implementations of \verb@while@ that have the same
behavior as the invocation of the function given above.

\chapter{\label{sec:ex-auction}Programming with Actors and Messages}

Here's an example that shows an application area for which Scala is
particularly well suited. Consider the task of implementing an
electronic auction service. We will use an Erlang-style actor process
model to implement the participants of the auction. Actors are
objects to which messages are sent. Every process has a ``mailbox'' of
its incoming messages which is represented as a queue. It can work
sequentially through the messages in its mailbox, or search for
messages matching some pattern.

For every traded item there is an auctioneer process that publishes
information about the traded item, that accepts offers from clients
and that communicates with the seller and winning bidder to close the
transaction. We present an overview of a simple implementation
here.

As a first step, we define the messages that are exchanged during an
auction. There are two abstract base classes (called {\em traits}):
\verb@AuctionMessage@ for messages from clients to the auction
service, and \verb@AuctionReply@ for replies from the service to the
clients.  These are defined as follows.
\begin{verbatim}
trait AuctionMessage;
case class 
  Offer(bid: Int, client: Actor),                     // make a bid
  Inquire(client: Actor) extends AuctionMessage;      // inquire status

trait AuctionReply;
case class
  Status(asked: Int, expiration: Date),               // asked sum, expiration date
  BestOffer(),                                        // yours is the best offer
  BeatenOffer(maxBid: Int),                           // offer beaten by maxBid
  AuctionConcluded(seller: Actor, client: Actor),     // auction concluded
  AuctionFailed(),                                    // failed with no bids
  AuctionOver() extends AuctionReply;                 // bidding is closed
\end{verbatim}

\begin{figure}[h]
\begin{verbatim}
class Auction(seller: Actor, minBid: Int, closing: Date) extends Actor() {
  val timeToShutdown = 36000000; // msec
  val bidIncrement = 10;
  def execute {
    var maxBid = minBid - bidIncrement;
    var maxBidder: Actor = _;
    var running = true;
    while (running) {
      receiveWithin ((closing.getTime() - new Date().getTime())) {
	case Offer(bid, client) =>
	  if (bid >= maxBid + bidIncrement) { 
            if (maxBid >= minBid) maxBidder send BeatenOffer(bid);
            maxBid = bid;
            maxBidder = client;
            client send BestOffer();
          } else {
            client send BeatenOffer(maxBid);
          }

	case Inquire(client) =>
	  client send Status(maxBid, closing);

	case TIMEOUT() =>
	  if (maxBid >= minBid) {
	    val reply = AuctionConcluded(seller, maxBidder);
	    maxBidder send reply;
	    seller send reply;
	  } else {
	    seller send AuctionFailed();
          }
          receiveWithin(timeToShutdown) {
            case Offer(_, client) => client send AuctionOver()
            case TIMEOUT() => running = false;
          }
      }
    }
  }
}
\end{verbatim}
\caption{\label{fig:simple-auction}Implementation of an Auction Service}
\end{figure}

For each base class, there are a number of {\em case classes} which
define the format of particular messages in the class. These messages
might well be ultimately implemented as small XML documents. We expect
automatic tools to exist that convert between XML documents and
internal data structures like the ones defined above.

Figure~\ref{fig:simple-auction} presents a Scala implementation of a
class \verb@Auction@ for auction processes that coordinate the bidding
on one item. Objects of this class are created by indicating
\begin{itemize}
\item
a seller process which needs to be notified when the auction is over,
\item
a minimal bid,
\item
the date when the auction is to be closed.
\end{itemize}
The process behavior is defined by its \verb@run@ method. That method
repeatedly selects (using \verb@receiveWithin@) a message and reacts to it,
until the auction is closed, which is signalled by a \verb@TIMEOUT@
message. Before finally stopping, it stays active for another period
determined by the \verb@timeToShutdown@ constant and replies to
further offers that the auction is closed.  Here are some further
explanations of the constructs used in this program:
\begin{itemize}
\item
The \verb@receiveWithin@ method of class \verb@Actor@ takes as
parameters a time span given in milliseconds and a function that
processes messages in the mailbox. The function is given by a sequence
of cases that each specify a pattern and an action to perform for
messages matching the pattern. The \verb@receiveWithin@ method selects
the first message in the mailbox which matches one of these patterns
and applies the corresponding action to it.
\item
The last case of \verb@receiveWithin@ is guarded by a
\verb@TIMEOUT()@ pattern. If no other messages are received in the meantime, this
pattern is triggered after the time span which is passed as argument
to the enclosing \verb@receiveWithin@ method. \verb@TIMEOUT()@ is a
particular instance of class \verb@Message@, which is triggered by the
\verb@Actor@ implementation itself.
\item
Reply messages are sent using syntax of the form
\verb@destination send SomeMessage@. \verb@send@ is used here as a
binary operator with a process and a message as arguments. This is
equivalent in Scala to the method call
\verb@destination.send(SomeMessage)@, i.e. the invocation of
the \verb@send@ of the destination process with the given message as
parameter.
\end{itemize}
The preceding discussion gave a flavor of distributed programming in
Scala. It might seem that Scala has a rich set of language constructs
that support actor processes, message sending and receiving,
programming with timeouts, etc. In fact, the opposite is true. All the
constructs discussed above are offered as methods in the library class
\verb@Actor@. That class is itself implemented in Scala, based on the underlying 
thread model of the host language (e.g. Java, or .NET).
The implementation of all features of class \verb@Actor@ used here is
given in Section~\ref{sec:actors}.

The advantages of this approach are relative simplicity of the core
language and flexibility for library designers. Because the core
language need not specify details of high-level process communication,
it can be kept simpler and more general. Because the particular model
of messages in a mailbox is a library module, it can be freely
modified if a different model is needed in some applications.  The
approach requires however that the core language is expressive enough
to provide the necessary language abstractions in a convenient
way. Scala has been designed with this in mind; one of its major
design goals was that it should be flexible enough to act as a
convenient host language for domain specific languages implemented by
library modules. For instance, the actor communication constructs
presented above can be regarded as one such domain specific language,
which conceptually extends the Scala core.

\chapter{Simple Functions}

The previous examples gave an impression of what can be done with
Scala.  We now introduce its constructs one by one in a more
systematic fashion. We start with the smallest level, expressions and
functions. 

A Scala system comes with an interpreter which can be seen as a
glorified calculator. A user interacts with the calculator by tying in
expressions and obtaining the results of their evaluation. Example:


\chapter{Classes and Objects}
\label{chap:classes}

Scala does not have a built-in type of rational numbers, but it is
easy to define one, using a class. Here's a possible
implementation.

\begin{verbatim}
class Rational(n: Int, d: Int) with {
  private def gcd(x: Int, y: Int): Int = {
    if (x == 0) y
    else if (x < 0) gcd(-x, y)
    else if (y < 0) -gcd(x, -y)
    else gcd(y % x, x);
  }
  private val g = gcd(n, d);

  val numer: Int = n/g;
  val denom: Int = d/g;
  def +(that: Rational) =
    new Rational(numer * that.denom + that.numer * denom, denom * that.denom);
  def -(that: Rational) =
    new Rational(numer * that.denom - that.numer * denom, denom * that.denom);
  def *(that: Rational) =
    new Rational(numer * that.numer, denom * that.denom);
  def /(that: Rational) =
    new Rational(numer * that.denom, denom * that.numer);
}
\end{verbatim}
This defines \verb@Rational@ as a class which takes two constructor arguments
\verb@n@ and \verb@d@, containing the number's numerator and denominator.
The class provides fields which return the number's numerator and
denominator as well as methods for arithmetic over rational numbers.
Each arithmetic method takes as parameter the right operand of the
operation. The left operand of the operation is always the rational
number of which the method is a member.

\paragraph{Private members.}
The implementation of rational numbers defines a private method
\verb@gcd@ which computes the greatest common denominator of two
integers, as well as a private field \verb@g@ which contains the
\verb@gcd@ of the constructor arguments. These members are inaccessible
outside class \verb@Rational@. They are used in the implementation of
the class to eliminate common factors in the constructor arguments in
order to ensure that nominator and denominator are always in
normalized form.

\paragraph{Creating and Accessing Objects.}
As an example of how rational numbers can be used, here's a program
that prints the sum of all numbers $1/i$ where $i$ ranges from 1 to 10.
\begin{verbatim}
var i = 1;
var x = Rational(0, 1);
while (i <= 10) {
  x = x + Rational(1,i);
  i = i + 1;
}
System.out.println(x.numer + "/" + x.denom);
\end{verbatim}

\paragraph{Inheritance and Overriding.}
Every class in Scala has a superclass which it extends.
Excepted is only the root class \verb@Object@, which does not have a
superclass, and which is indirectly extended by every other class.
If a class does not mention a superclass in its definition, the root
class \verb@Object@ is implicitly assumed. For instance, class
\verb@Rational@ could equivalently be defined as
\begin{verbatim}
class Rational(n: Int, d: Int) extends Object with {
  ... // as before
}
\end{verbatim}
A class inherits all members from its superclass. It may also redefine
(or: {\em override}) some inherited members. For instance, class
\verb@Object@ defines
a method
\verb@toString@ which returns a representation of the object as a string:
\begin{verbatim}
class Object {
  ...
  def toString(): String = ...
}
\end{verbatim}
The implementation of \verb@toString@ in \verb@Object@
forms a string consisting of the object's class name and a number. It
makes sense to redefine this method for objects that are rational
numbers in order to get a more useful behavior:
\begin{verbatim}
class Rational(n: Int, d: Int) extends Object with {
  ... // as before
  override def toString() = numer + "/" + denom;
}
\end{verbatim}
Note that, unlike in Java, redefining definitions need to be preceded
by an \verb@override@ modifier.

If class $A$ extends class $B$, then objects of type $A$ may be used
whereever objects of type $B$ are expected. We say in this case that
type $A$ {\em conforms} to type $B$.  For instance, \verb@Rational@
conforms to \verb@Object@, so it is legal to assign a \verb@Rational@
value to a variable of type \verb@Object@:
\begin{verbatim}
var x: Object = new Rational(1,2);
\end{verbatim}

\paragraph{Parameterless Methods.}
%Also unlike in Java, methods in Scala do not necessarily take a
%parameter list. An example is \verb@toString@; the method is invoked
%by simply mentioning its name. For instance:
%\begin{verbatim}
%val r = new Rational(1,2);
%System.out.println(r.toString());	// prints``1/2''
%\end{verbatim}
Also unlike in Java, methods in Scala do not necessarily take a
parameter list. An example is \verb@square@; the method is invoked by
simply mentioning its name. For instance:
\begin{verbatim}
class Rational(n: Int, d: Int) extends Object with {
  ... // as before
  def square = Rational(numer*numer, denom*denom);
}
val r = new Rational(3,4);
System.out.println(r.square);		// prints``9/16''
\end{verbatim}
That is, parameterless methods are accessed just as value fields such
as \verb@numer@ are. The difference between values and parameterless
methods lies in their definition. The right-hand side of a value is
evaluated when the object is created, and the value does not change
afterwards. A right-hand side of a parameterless method, on the other
hand, is evaluated each time the method is called.  The uniform access
of fields and parameterless methods gives increased flexibility for
the implementer of a class. Often, a field in one version of a class
becomes a computed value in the next version. Uniform access ensures
that clients do not have to be rewritten because of that change.

\paragraph{Abstract Methods.}
Classes can also omit some of the definitions of their members.  As an
example, consider the following class \verb@Ord@ which provides the
comparison operators \verb@<, >, <=, >=@.
%\begin{verbatim}
%abstract class Ord with {
%  abstract def <(that: this);
%  def <=(that: this)  =  this < that || this == that;
%  def >(that: this)  =  that < this;
%  def >=(that: this)  =  that <= this;
%}
%\end{verbatim}
\begin{verbatim}
abstract class Ord with {
  def <(that: this): Boolean;
  def <=(that: this)  =  this < that || this == that;
  def >(that: this)  =  that < this;
  def >=(that: this)  =  that <= this;
}
\end{verbatim}
Since we want to leave open which objects are compared, we are unable
to give an implementation for the \verb@<@ method. However, once
\verb@<@ is given, we can define the other three comparison operators
in terms of \verb@<@ and the equality test \verb@==@ (which is defined
in class \verb@Object@).  This is expressed by having in \verb@Ord@ an
{\em abstract} method \verb@<@ to which the implementations of the
other methods refer.

\paragraph{Self References.} The name \verb@this@ refers in this class
to the current object. The type of \verb@this@ is also called
\verb@this@ (generally, every name in Scala can have a definition as a
term and another one as a type).  When used as a type, \verb@this@
refers to the type of the current object. This type is always
compatible with the class being defined (\verb@Ord@ in this case).

\paragraph{Mixin Composition.}
We can now define a class of \verb@Rational@ numbers that
support comparison operators.
\begin{verbatim}
final class OrderedRational(n: Int, d: Int)
 extends Rational(n, d) with Ord with {
  override def ==(that: OrderedRational) =
    numer == that.numer && denom == that.denom;
  def <(that: OrderedRational): Boolean =
    numer * that.denom < that.numer * denom;
}
\end{verbatim}
Class \verb@OrderedRational@ redefines method \verb@==@, which is
defined as reference equality in class \verb@Object@. It also
implements the abstract method \verb@<@ from class \verb@Ord@.  In
addition, it inherits all members of class \verb@Rational@ and all
non-abstract members of class \verb@Ord@. The implementations of
\verb@==@ and \verb@<@ replace the definition of \verb@==@ in class
\verb@Object@ and the abstract declaration of \verb@<@ in class
\verb@Ord@. The other inherited comparsion methods then refer to this
implementation in their body.

The clause ``\verb@Rational(d, d) with Ord@'' is an instance of {\em
mixin composition}. It describes a template for an object that is
compatible with both \verb@Rational@ and \verb@Ord@ and that contains
all members of either class. \verb@Rational@ is called the {\em
superclass} of \verb@OrderedRational@ while \verb@Ord@ is called a
{\em mixin class}. The type of this template is the {\em compound
type} ``\verb@Rational with Ord@''.

On the surface, mixin composition looks much like multiple
inheritance. The difference between the two becomes apparent if we
look at superclasses of inherited classes. With multiple inheritance,
both \verb@Rational@ and \verb@Ord@ would contribute a superclass
\verb@Object@ to the template. We therefore have to answer some
tricky questions, such as whether members of \verb@Object@ are present
once or twice and whether the initializer of \verb@Object@ is called
once or twice. Mixin composition avoids these complications.  In the
mixin composition \verb@Rational with Ord@, class
\verb@Rational@ is treated as actual superclass of class \verb@Ord@.
A mixin composition \verb@C with M@ is well-formed as long as the
first operand \verb@C@ conforms to the declared superclass of the
second operand \verb@M@. This holds in our example, because
\verb@Rational@ conforms to \verb@Object@. In a sense, mixin composition
amounts to overriding the superclass of a class.

\paragraph{Final Classes.}
Note that class \verb@OrderedRational@ was defined
\verb@final@. This means that no classes extending \verb@OrderedRational@
may be defined in other parts of the program.
%Within final classes the
%type \verb@this@ is an alias of the defined class itself. Therefore,
%we could define our \verb@<@ method with an argument of type
%\verb@OrderedRational@ as a well-formed implementation of the abstract class
%\verb@less(that: this)@ in class \verb@Ord@.


\chapter{Generic Types and Methods}

Classes in Scala can have type parameters. We demonstrate the use of
type parameters with iterators as an example. An iterator is an object
which traverses a sequence of values, using two abstract methods.
\begin{verbatim}
abstract class Iterator[a] with {
  def hasNext: Boolean;
  def next: a;
\end{verbatim}
Method \verb@next@ returns succesive elements.  Method \verb@hasNext@
indicates whether there are still more elements to be returned by
\verb@next@. The type of elements returned by an iterator is
arbitrary. We express that by giving the class \verb@Iterator@ the
type parameter \verb@a@. Type parameters are written in square
brackets, in contrast to normal value parameters, which are written in
parentheses.  Iterators also support other methods, which are
explained in the following.

Method \verb@foreach@ applies a procedure (i.e. a function
returning \verb@Unit@ to each element returned by the iterator:
\begin{verbatim}
  def foreach(f: (a)Unit): Unit =
    while (hasNext) { f(next) }
\end{verbatim}

Method \verb@append@ constructs an iterator which resumes with the
given iterator \verb@it@ after the current iterator has finished.
\begin{verbatim}
  def append(that: Iterator[a]): Iterator[a] = new Iterator[a] with {
    def hasNext = outer.hasNext || that.hasNext;
    def next = if (outer.hasNext) outer.next else that.next;
  }
\end{verbatim}
The terms \verb@outer.next@ and \verb@outer.hasNext@ in the definition
of \verb@append@ call the corresponding methods as they are defined in
the enclosing \verb@Iterator@ class. Generally, an
\verb@outer@ prefix in a selection indicates an identifer that is
visible immediately outside the current class or template. If the
\verb@outer@ prefix would have been missing,
\verb@hasNext@ and \verb@next@ would have
called recursively the methods being defined in the iterator
constructed by \verb@append@, which is not what we want.

Method \verb@filter@ constructs an iterator which returns all elements
of the original iterator that satisfy a criterion \verb@p@.
\begin{verbatim}
  def filter(p: (a)Boolean) = new Iterator[a] with {
    private class Cell[T](elem_: T) with { def elem = elem_; }
    private var head: Cell[a] = null;
    private var isAhead = False;
    def hasNext: Boolean =
      if (isAhead) True
      else if (outer.hasNext) {
        head = Cell(outer.next); isAhead = p(head.elem); hasNext }
      else False;
    def next: a =
      if (hasNext) { isAhead = False; head.elem }
      else error("next on empty iterator");
  }
\end{verbatim}
Method \verb@map@ constructs an iterator which returns all elements of
the original iterator transformed by a given function \verb@f@.
\begin{verbatim}
  def map[b](f: (a)b) = new Iterator[b] with {
    def hasNext: Boolean = outer.hasNext;
    def next: b = f(outer.next);
  }
\end{verbatim}
The return type of the transformation function \verb@f@ is
arbitrary. This is expressed by a type parameter \verb@b@ in the
signature of method \verb@map@, which represents the return type.
We also say, \verb@map@ is a {\em polymorphic} function.

Method \verb@flatMap@ is like method \verb@map@, except that the
transformation function \verb@f@ now returns an iterator.
The result of \verb@flatMap@ is the iterator resulting from appending
together all iterators returned from successive calls of \verb@f@.
\begin{verbatim}
    private var cur: Iterator[b] = new EmptyIterator[b];
    def hasNext: Boolean =
      if (cur.hasNext) True
      else if (outer.hasNext) { cur = f(outer.next); hasNext }
      else False;
    def next: b =
      if (cur.hasNext) cur.next
      else if (outer.hasNext) { cur = f(outer.next); next }
      else error("next on empty iterator");
  }
\end{verbatim}
Finally, method \verb@zip@ takes another iterator and
returns an iterator consisting of pairs of corresponding elements
returned by the two iterators.
\begin{verbatim}
  def zip[b](that: Iterator[b]) = new Iterator[(a, b)] with {
    def hasNext = outer.hasNext && that.hasNext;
    def next = (outer.next, that.next);
  }
} //end iterator;
\end{verbatim}
Concrete iterators need to provide implementations for the two
abstract methods \verb@next@ and \verb@hasNext@ in class
\verb@Iterator@. The simplest iterator is \verb@EmptyIterator@
which always returns an empty sequence:
\begin{verbatim}
class EmptyIterator[a] extends Iterator[a] with {
  def hasNext = False;
  def next: a = error("next on empty iterator");
}
\end{verbatim}
A more interesting iterator enumerates all elements of an array.
This iterator is formulated here as a polymorphic function. It could
have also been written as a class, like \verb@EmptyIterator@. The
difference between the two formulation is that classes also define new
types, whereas functions do not.
\begin{verbatim}
def arrayIterator[a](xs: Array[a]) = new Iterator[a] with {
  private var i = 0;
  def hasNext: Boolean =
    i < xs.length;
  def next: a =
    if (i < xs.length) { val x = xs(i) ; i = i + 1 ; x }
    else error("next on empty iterator");
}
\end{verbatim}
Another iterator enumerates an integer interval:
\begin{verbatim}
def range(lo: Int, hi: Int) = new Iterator[Int] with {
  private var i = lo;
  def hasNext: Boolean =
    i <= hi;
  def next: Int =
    if (i <= hi) { i = i + 1 ; i - 1 }
    else error("next on empty iterator");
}
\end{verbatim}
%In fact, enumerating integer intervals is so common that it is
%supported by a method
%\begin{verbatim}
%def to(hi: Int): Iterator[Int]
%\end{verbatim}
%in class \verb@Int@. Hence, one could also write \verb@l to h@ instead of
%\verb@range(l, h)@.
All iterators seen so far terminate eventually. It is also possible to
define iterators that go on forever. For instance, the following
iterator returns successive integers from some start
value\footnote{Due to the finite representation of type \prog{Int},
numbers will wrap around at $2^31$.}.
\begin{verbatim}
def from(start: Int) = new Iterator[Int] with {
  private var last = start - 1;
  def hasNext = True;
  def next = { last = last + 1; last }
}
\end{verbatim}
Here are two examples how iterators are used. First, to print all
elements of an array \verb@xs: Array[Int]@, one can write:
\begin{verbatim}
  arrayIterator[Int](xs) foreach (x => System.out.println(x))
\end{verbatim}
Here, \verb@[Int]@ is a type argument clause, which matches the type
parameter clause \verb@[a]@ of function \verb@arrayIterator@. It
substitutes the formal argument \verb@Int@ for the formal argument
\verb@a@ in the type of the method that follows. Hence,
\verb@arrayIterator[a]@ is a function that takes an \verb@Array[Int]@
and that returns an \verb@Iterator[Int]@.

In this example, the formal type argument \verb@Int@ is redundant
since it could also have been inferred from the value \verb@xs@, which
is, after all, an array of \verb@Int@. The Scala compiler contains a
fairly powerful type inferencer which infers type arguments for
methods and constructors from the types of value arguments and the
expected return type. In our example, the \verb@[Int]@ clause can be
inferred, so that one can abbreviate to:
\begin{verbatim}
  arrayIterator(xs) foreach (x => System.out.println(x))
\end{verbatim}
%As a second example, consider the problem of finding the indices of
%all the elements in an array of \verb@Double@s greater than some
%\verb@limit@. The indices should be returned as an iterator.
%This is achieved by the following expression.
%\begin{verbatim}
%arrayIterator(xs)
%  .zip(from(0))
%  .filter(x, i => x > limit)
%  .map(x, i => i)
%\end{verbatim}
%The first line in this expression iterates through all array elements,
%the second lines pairs elements with their indices, the third line
%selects all value/index pairs where the value is greater than
%\verb@limit@, and the fourth line returns the index part of all
%selected pairs.

%Note that we have omitted the type arguments for the calls of
%\verb@arrayIterator@, \verb@zip@ and \verb@map@. These are all
%implicitly inserted by the type inferencer.

\chapter{\label{sec:for-notation}For-Comprehensions}

The last chapter has demonstrated that the use of higher-order
functions over sequences can lead to very concise programs. But
sometimes the level of abstraction required by these functions makes a
program hard to understand.

Here, Scala's \verb@for@ notation can help. For instance, say we are
given a sequence \verb@persons@ of persons with \verb@name@ and
\verb@age@ fields.  That sequence could be an array, or a list, or an
iterator, or some other type implementing the sequence abstraction
(this will be made more precise below). To print the names of all
persons in the sequence which are aged over 20, one writes:
\begin{verbatim}
for { val p <- persons; p.age > 20 } yield p.name
\end{verbatim}
This is equivalent to the following expression , which uses
higher-order functions \verb@filter@ and \verb@map@:
\begin{verbatim}
persons filter (p => p.age > 20) map (p => p.name)
\end{verbatim}
The for-expression looks a bit like a for-loop in imperative languages,
except that it constructs a list of the results of all iterations.

Generally, a for-comprehension is of the form
\begin{verbatim}
for ( s ) yield e
\end{verbatim}
(Instead of parentheses, braces may also be used.)
Here, \verb@s@ is a sequence of {\em generators} and {\em filters}.
\begin{itemize}
\item A {\em generator} is of the form \verb@val x <- e@,
where \verb@e@ is a list-valued expression. It binds \verb@x@ to
successive values in the list.
\item A {\em filter} is an expression \verb@f@ of type \verb@Boolean@.
It omits from consideration all bindings for which \verb@f@ is \verb@False@.
\end{itemize}
The sequence must start with a generator.
If there are several generators in a sequence, later generators vary
more rapidly than earlier ones.

Here are two examples that show how for-comprehensions are used.

First, given a positive integer \verb@n@, find all pairs of positive
integers
\verb@i@, \verb@j@, where \verb@1 <= j < i <= n@ such that \verb@i + j@ is prime.
\begin{verbatim}
for \={ \=val i <- range(1, n);
    \>  \>val j <- range(1, i-1);
    \>  \>isPrime(i+j)
} yield (i, j)
\end{verbatim}

As second example, the scalar product of two vectors \verb@xs@ and
\verb@ys@ can now be written as
follows.
\begin{verbatim}
  sum (for { val (x, y) <- xs zip ys } yield x * y)
\end{verbatim}
The for-notation is essentially equivalent to common operations of
database query languages.  For instance, say we are given a book
database \verb@books@, represented as a list of books, where
\verb@Book@ is defined as follows.
\begin{verbatim}
abstract class Book with {
  val title: String;
  val authors: List[String]
}
\end{verbatim}
\begin{verbatim}
val books: List[Book] = [
  new Book with {
    val title = "Structure and Interpretation of Computer Programs";
    val authors = ["Abelson, Harald", "Sussman, Gerald J."];
  },
  new Book with {
    val title = "Principles of Compiler Design";
    val authors = ["Aho, Alfred", "Ullman, Jeffrey"];
  },
  new Book with {
    val title = "Programming in Modula-2";
    val authors = ["Wirth, Niklaus"];
  }
];
\end{verbatim}
Then, to find the titles of all books whose author's last name is ``Ullman'':
\begin{verbatim}
for { val b <- books; val a <- b.authors; a startsWith "Ullman"
} yield b.title
\end{verbatim}
(Here, \verb@startsWith@ is a method in \verb@java.lang.String@).  Or,
to find the titles of all books that have the string ``Program'' in
their title:
\begin{verbatim}
for { val b <- books; (b.title indexOf "Program") >= 0
} yield b.title
\end{verbatim}
Or, to find the names of all authors that have written at least two
books in the database.
\begin{verbatim}
for { \=val b1 <- books;
      \>val b2 <- books;
      \>b1 != b2;
      \>val a1 <- b1.authors;
      \>val a2 <- b2.authors;
      \>a1 == a2 } yield a1
\end{verbatim}
The last solution is not yet perfect, because authors will appear
several times in the list of results.  We still need to remove
duplicate authors from result lists.  This can be achieved with the
following function.
\begin{verbatim}
def removeDuplicates[a](xs: List[a]): List[a] =
  if (xs.isEmpty) xs
  else xs.head :: removeDuplicates(xs.tail filter (x => x != xs.head));
\end{verbatim}
The last expression can be equivalently expressed as follows.
\begin{verbatim}
xs.head :: removeDuplicates(for (val x <- xs.tail; x != xs.head) yield x)
\end{verbatim}

\subsection*{Translation of \prog{for}}

Every for-comprehensions can be expressed in terms of the three
higher-order functions \verb@map@, \verb@flatMap@ and \verb@filter@.
Here is the translation scheme, which is also used by the Scala compiler.
\begin{itemize}
\item
A simple for-comprehension
\begin{verbatim}
for (val x <- e) yield e'
\end{verbatim}
is translated to
\begin{verbatim}
e.map(x => e')
\end{verbatim}
\item
A for-comprehension
\begin{verbatim}
for (val x <- e; f; s) yield e'
\end{verbatim}
where \verb@f@ is a filter and \verb@s@ is a (possibly empty)
sequence of generators or filters
is translated to
\begin{verbatim}
for (val x <- e.filter(x => f); s) yield e'
\end{verbatim}
and then translation continues with the latter expression.
\item
A for-comprehension
\begin{verbatim}
for (val x <- e; y <- e'; s) yield e''
\end{verbatim}
where \verb@s@ is a (possibly empty)
sequence of generators or filters
is translated to
\begin{verbatim}
e.flatMap(x => for (y <- e'; s) yield e'')
\end{verbatim}
and then translation continues with the latter expression.
\end{itemize}
For instance, taking our "pairs of integers whose sum is prime" example:
\begin{verbatim}
for \= { \= val i <- range(1, n);
    \>   \> val j <- range(1, i-1);
    \>   \> isPrime(i+j)
} yield (i, j)
\end{verbatim}
Here is what we get when we translate this expression:
\begin{verbatim}
range(1, n)
  .flatMap(i =>
    range(1, i-1)
      .filter(j => isPrime(i+j))
      .map(j => (i, j)))
\end{verbatim}

\exercise
Define the following function in terms of \verb@for@.
\begin{verbatim}
def concat(xss: List[List[a]]): List[a] =
  (xss foldr []) { xs, ys => xs ::: ys }
\end{verbatim}
\exercise
Translate
\begin{verbatim}
for { val b <- books; val a <- b.authors; a startsWith "Bird" } yield b.title
for { val b <- books; (b.title indexOf "Program") >= 0 } yield b.title
\end{verbatim}
to higher-order functions.

We have seen that the for-translation only relies on the presence of
methods \verb@map@,
\verb@flatMap@, and \verb@filter@.
This gives programmers the possibility to have for-syntax for
other types as well -- one only needs to define \verb@map@,
\verb@flatMap@, and \verb@filter@ for these types.
That's also why we were able to define \verb@for@ at the same time for
arrays, iterators, and lists -- all these types have the required
three methods \verb@map@,\verb@flatMap@, and \verb@filter@ as members.
Of course, it is also possible for users and librariy designers to
define other types with these methods. There are many examples where
this is useful: Databases, XML trees, optional values. We will see in
Chapter~\ref{sec:parsers-results} how for-comprehesnions can be used in the
definition of parsers for context-free grammars that construct
abstract syntax trees.

\chapter{\label{sec:simple-examples}Pattern Matching}

\todo{Complete}

Consider binary trees whose leafs contain integer arguments. This can
be described by a class for trees, with subclasses for leafs and
branch nodes:
\begin{verbatim}
abstract class Tree;
case class Branch(left: Tree, right: Tree) extends Tree;
case class Leaf(x: Int) extends Tree;
\end{verbatim}
Note that the class \verb@Tree@ is not followed by an extends
clause or a body. This defines \verb@Tree@ to be an empty
subclass of \verb@Object@, as if we had written
\begin{verbatim}
class Tree extends Object with {}
\end{verbatim}
Note also that the two subclasses of \verb@Tree@ have a \verb@case@
modifier.  That modifier has two effects. First, it lets us construct
values of a case class by simply calling the constructor, without
needing a preceding \verb@new@. Example:
\begin{verbatim}
val tree1 = Branch(Branch(Leaf(1), Leaf(2)), Branch(Leaf(3), Leaf(4)))
\end{verbatim}
Second, it lets us use constructors for these classes in patterns, as
is illustrated in the following example.
\begin{verbatim}
def sumLeaves(t: Tree): Int = t match {
  case Branch(l, r) => sumLeaves(l) + sumLeaves(r)
  case Leaf(x) => x
}
\end{verbatim}
The function \verb@sumLeaves@ sums up all the integer values in the
leaves of a given tree \verb@t@. It is is implemented by calling the
\verb@match@ method of \verb@t@ with a {\em choice expression} as
argument (\verb@match@ is a predefined method in class \verb@Object@).
The choice expression consists of two cases which both
relate a pattern with an expression. The pattern of the first case,
\verb@Branch(l, r)@ matches all instances of class \verb@Branch@
and binds the {\em pattern variables} \verb@l@ and \verb@r@ to the
constructor arguments, i.e.\ the left and right subtrees of the
branch.  Pattern variables always start with a lower case letter; to
avoid ambiguities, constructors in patterns should start with an upper
case letter.

The effect of the choice expression is to select the first alternative
whose pattern matches the given select value, and to evaluate the body
of this alternative in a context where pattern variables are bound to
corresponding parts of the selector. For instance, the application
\verb@sumLeaves(tree1)@ would select the first alternative with the
\verb@Branch(l,r)@ pattern, and would evaluate the expression
\verb@sumLeaves(l) + sumLeaves(r)@ with bindings
\begin{verbatim}
l = Branch(Leaf(1), Leaf(2)), r = Branch(Leaf(3), Leaf(4)).
\end{verbatim}
As another example, consider the following class
\begin{verbatim}
abstract final class Option[a];
case class None[a] extends Option[a];
case class Some[a](item: a) extends Option[a];
\end{verbatim}
...

%\todo{Several simple and intermediate examples needed}.

\begin{verbatim}
def find[a,b](it: Iterator[(a, b)], x: a): Option[b] = {
  var result: Option[b] = None;
  while (it.hasNext && result == None) {
    val (x1, y) = it.next;
    if (x == x1) result = Some(y)
  }
  result
}
find(xs, x) match {
  case Some(y) => System.out.println(y)
  case None => System.out.println("no match")
}
\end{verbatim}

\comment{


class MaxCounter with {
  var maxVal: Option[Int] = None;
  def set(x: Int) = maxVal match {
    case None => maxVal = Some(x)
    case Some(y) => maxVal = Some(Math.max(x, y))
  }
}
\end{verbatim}
}
\comment{
\begin{verbatim}
class Stream[a] = List[a]

module Stream with {
  def concat(xss: Stream[Stream[a]]): Stream[a] = {
    let result: Stream[a] = xss match {
      case [] => []
      case [] :: xss1 => concat(xss1)
      case (x :: xs) :: xss1 => x :: concat(xs :: xss1)
    }
    result
  }
}
\end{verbatim}
}
\comment{
\chapter{Implementing Abstract Types: Search Trees}

This chapter presents unbalanced binary search trees, implemented in
three different styles: algebraic, object-oriented, and imperative.
In each case, a search tree package is seen as an implementation
of a class {\em MapStruct}.
\begin{verbatim}
abstract class MapStruct[kt, vt] with {
  abstract type Map extends (kt)vt with {
    def apply(key: kt): vt;
    def extend(key: kt, value: vt): Map;
    def remove(key: kt): Map;
    def domain: Stream[kt];
    def range: Stream[vt];
  }
  def empty: Map;
}
\end{verbatim}
The \verb@MapStruct@ class is parameterized with a type of keys
\verb@kt@ and a type of values \verb@vt@. It
specifies an abstract type \verb@Map@ and an abstract value
\verb@empty@, which represents empty maps.  Every implementation
\verb@Map@ needs to conform to that abstract type, which
extends the function type \verb@(kt)vt@
with four new
methods. The method \verb@domain@ yields a stream that enumerates the
map's domain, i.e. the set of keys that are mapped to non-null values.
The method \verb@range@ yields a stream that enumerates the function's
range, i.e.\ the values obtained by applying the function to arguments
in its domain.  The method
\verb@extend@ extends the map with a given key/value binding, whereas
\verb@remove@ removes a given key from the map's domain. Both
methods yield a new map value as result, which has the same
representation as the receiver object.

\begin{figure}[t]
\begin{verbatim}
class AlgBinTree[kt extends Ord, vt] extends MapStruct[kt, vt] with {
  private case
    Empty extends Map,
    Node(key: kt, value: vt, l: Map, r: Map) extends Map

  final class Map extends (kt)vt with {
    def apply(key: kt): vt = this match {
      case Empty => null
      case Node(k, v, l, r) =>
	if (key < k) l.apply(key)
	else if (key > k) r.apply(key)
	else v
    }

    def extend(key: kt, value: vt): Map = this match {
      case Empty => Node(k, v, Empty, Empty)
      case Node(k, v, l, r) =>
	if (key < k) Node(k, v, l.extend(key, value), r)
	else if (key > k) Node(k, v, l, r.extend(key, value))
	else Node(k, value, l, r)
    }

    def remove(key: kt): Map = this match {
      case Empty => Empty
      case Node(k, v, l, r) =>
	if (key < k) Node(k, v, l.remove(key), r)
	else if (key > k) Node(k, v, l, r.remove(key))
	else if (l == Empty) r
	else if (r == Empty) l
	else {
	  val midKey = r.domain.head
	  Node(midKey, r.apply(midKey), l, r.remove(midKey))
	}
    }

    def domain: Stream[kt] = this match {
      case Empty => []
      case Node(k, v, l, r) => Stream.concat([l.domain, [k], r.domain])
    }
    def range: Stream[vt] = this match {
      case Empty => []
      case Node(k, v, l, r) => Stream.concat([l.range, [v], r.range])
    }
  }
  def empty: Map = Empty
}
\end{verbatim}
\caption{\label{fig:algbintree}Algebraic implementation of binary
search trees}
\end{figure}
We now present three implementations of \verb@Map@, which are all
based on binary search trees. The \verb@apply@ method of a map is
implemented in each case by the usual search function over binary
trees, which compares a given key with the key stored in the topmost
tree node, and depending on the result of the comparison, searches the
left or the right hand sub-tree. The type of keys must implement the
\verb@Ord@ class, which contains comparison methods
(see Chapter~\ref{chap:classes} for a definition of class \verb@Ord@).

The first implementation, \verb@AlgBinTree@, is given in
Figure~\ref{fig:algbintree}. It represents a map with a
data type \verb@Map@ with two cases, \verb@Empty@ and \verb@Node@.

Every method of \verb@AlgBinTree[kt, vt].Map@ performs a pattern
match on the value of
\verb@this@ using the \verb@match@ method which is defined as postfix
function application in class \verb@Object@ (\sref{sec:class-object}).

The functions \verb@domain@ and \verb@range@ return their results as
lazily constructed lists. The \verb@Stream@ class is an alias of
\verb@List@ which should be used to indicate the fact that its values
are constructed lazily.

\begin{figure}[thb]
\begin{verbatim}
class OOBinTree[kt extends Ord, vt] extends MapStruct[kt, vt] with {
  abstract class Map extends (kt)vt with {
    def apply(key: kt): v
    def extend(key: kt, value: vt): Map
    def remove(key: kt): Map
    def domain: Stream[kt]
    def range: Stream[vt]
  }
  module empty extends Map with {
    def apply(key: kt) = null
    def extend(key: kt, value: vt) = Node(key, value, empty, empty)
    def remove(key: kt) = empty
    def domain = []
    def range = []
  }
  private class Node(k: kt, v: vt, l: Map, r: Map) extends Map with {
    def apply(key: kt): vt =
      if (key < k) l.apply(key)
      else if (key > k) r.apply(key)
      else v
    def extend(key: kt, value: vt): Map =
      if (key < k) Node(k, v, l.extend(key, value), r)
      else if (key > k) Node(k, v, l, r.extend(key, value))
      else Node(k, value, l, r)
    def remove(key: kt): Map =
      if (key < k) Node(k, v, l.remove(key), r)
      else if (key > k) Node(k, v, l, r.remove(key))
      else if (l == empty) r
      else if (r == empty) l
      else {
	val midKey = r.domain.head
	Node(midKey, r(midKey), l, r.remove(midKey))
      }
    def domain: Stream[kt] = Stream.concat([l.domain, [k], r.domain]	)
    def range: Stream[vt] = Stream.concat([l.range, [v], r.range])
  }
}
\end{verbatim}
\caption{\label{fig:oobintree}Object-oriented implementation of binary
search trees}
\end{figure}

The second implementation of maps is given in
Figure~\ref{fig:oobintree}.  Class \verb@OOBinTree@ implements the
type \verb@Map@ with a module \verb@empty@ and a class
\verb@Node@, which define the behavior of empty and non-empty trees,
respectively.

Note the different nesting structure of \verb@AlgBinTree@ and
\verb@OOBinTree@. In the former, all methods form part of the base
class \verb@Map@. The different behavior of empty and non-empty trees
is expressed using a pattern match on the tree itself. In the
latter, each subclass of \verb@Map@ defines its own set of
methods, which override the methods in the base class. The pattern
matches of the algebraic implementation have been replaced by the
dynamic binding that comes with method overriding.

Which of the two schemes is preferable depends to a large degree on
which extensions of the type are anticipated. If the type is later
extended with a new alternative, it is best to keep methods in each
alternative, the way it was done in \verb@OOBinTree@.  On the other
hand, if the type is extended with additional methods, then it is
preferable to keep only one implementation of methods and to rely on
pattern matching, since this way existing subclasses need not be
modified.

\begin{figure}
\begin{verbatim}
class MutBinTree[kt extends Ord, vt] extends MapStruct[kt, vt] with {
  class Map(key: kt, value: vt) extends (kt)vt with {
    val k = key
    var v = value
    var l = empty, r = empty

    def apply(key: kt): vt =
      if (this eq empty) null
      else if (key < k) l.apply(key)
      else if (key > k) r.apply(key)
      else v

    def extend(key: kt, value: vt): Map =
      if (this eq empty) Map(key, value)
      else {
	if (key < k) l = l.extend(key, value)
	else if (key > k) r = r.extend(key, value)
	else v = value
	this
      }

    def remove(key: kt): Map =
      if (this eq empty) this
      else if (key < k) { l = l.remove(key) ; this }
      else if (key > k) { r = r.remove(key) ; this }
      else if (l eq empty) r
      else if (r eq empty) l
      else {
	var mid = r
	while (!(mid.l eq empty)) { mid = mid.l }
	mid.r = r.remove(mid.k)
	mid.l = l
	mid
      }

    def domain: Stream[kt] = Stream.concat([l.domain, [k], r.domain])
    def range: Stream[vt] = Stream.concat([l.range, [v], r.range])
  }
  let empty = new Map(null, null)
}
\end{verbatim}
\caption{\label{fig:impbintree}Side-effecting implementation of binary
search trees}
\end{figure}

The two versions of binary trees presented so far are {\em
persistent}, in the sense that maps are values that cannot be changed
by side effects. By contrast, in the next implementation of binary
trees, the implementations of \verb@extend@ and
\verb@remove@ do have an effect on the state of their receiver
object. This corresponds to the way binary trees are usually
implemented in imperative languages. The new implementation can lead
to some savings in computing time and memory allocation, but care is
required not to use the original tree after it has been modified by a
side-effecting operation.

In this implementation, \verb@value@, \verb@l@ and \verb@r@ are
variables that can be affected by method calls.  The
class \verb@MutBinTree[kt, vt].Map@ takes two instance parameters
which define the \verb@key@ value and the initial value of the
\verb@value@ variable. Empty trees are represented by a
value \verb@empty@, which has \verb@null@ (signifying undefined) in
both its key and value fields. Note that this value needs to be
defined lazily using \verb@let@ since its definition involves the
creation of a
\verb@Map@ object,
which accesses \verb@empty@ recursively as part of its initialization.
All methods test first whether the current tree is empty using the
reference equality operator \verb@eq@ (\sref{sec:class-object}).

As a program using the \verb@MapStruct@ abstraction, consider a function
which creates a map from strings to integers and then applies it to a
key string:
\begin{verbatim}
def mapTest(def mapImpl: MapStruct[String, Int]): Int = {
  val map: mapImpl.Map = mapImpl.empty.extend("ab", 1).extend("bx", 3)
  val x = map("ab")             // returns 1
}
\end{verbatim}
The function is parameterized with the particular implementation of
\verb@MapStruct@. It can be applied to any one of the three implementations
described above. E.g.:
\begin{verbatim}
mapTest(AlgBinTree[String, Int])
mapTest(OOBinTree[String, Int])
mapTest(MutBinTree[String, Int])
\end{verbatim}
}
\chapter{Programming with Higher-Order Functions: Combinator Parsing}

In this chapter we describe how to write combinator parsers in
Scala. Such parsers are constructed from predefined higher-order
functions, so called parser combinators, that closely model the
constructions of an EBNF grammar \cite{ebnf}.

As running example, we consider parsers for arithmetic expressions
described by the following context-free grammar.
\bda{p{3cm}cp{10cm}}
letter &::=& /* all letters */ \\
digit  &::=& /* all digits */ \\[0.5em]
ident  &::=& letter \{letter $|$ digit \}\\
number &::=& digit \{digit\}\\[0.5em]

expr &::=& expr1 \{`+' expr1 $|$ `$-$' expr1\}\\
expr1 &::=& expr2 \{`*' expr2 $|$ `/' expr2\}\\
expr2 &::=& ident $|$ number $|$ `(' expr `)'
\eda

\section{Simple Combinator Parsing}

In this section we will only be concerned with the task of recognizing
input strings, not with processing them. So we can describe parsers
by the sets of input strings they accept.  There are two
fundamental operators over parsers:
\verb@&&&@ expresses the sequential composition of a parser with
another, while \verb@|||@ expresses an alternative. These operations
will both be defined as methods of a \verb@Parser@ class.  We will
also define constructors for the following primitive parsers:

\begin{quote}\begin{tabular}{ll}
\verb@empty@	& The parser that accepts the empty string
\\
\verb@fail@     & The parser that accepts no string
\\
\verb@chr@      & The parser that accepts any character.
\\
\verb@chr(c: Char)@
		& The parser that accepts the single-character string ``$c$''.
\\
\verb@chrWith(p: (Char)Boolean)@
		& The parser that accepts single-character strings
                  ``$c$'' \\
	        & for which $p(c)$ is true.
\end{tabular}\end{quote}

There are also the two higher-order parser combinators \verb@opt@,
expressing optionality and \verb@rep@, expressing repetition.
For any parser $p$, \verb@opt($p$)@ yields a parser that
accepts the strings accepted by $p$ or else the empty string, while
\verb@rep($p$)@ accepts arbitrary sequences of the strings accepted by
$p$. In EBNF, \verb@opt($p$)@ corresponds to $[p]$ and \verb@rep($p$)@
corresponds to $\{p\}$.

The central idea of parser combinators is that parsers can be produced
by a straightforward rewrite of the grammar, replacing \verb@::=@ with
\verb@=@, sequencing with
\verb@&&&@, choice
\verb@|@ with \verb@|||@, repetition \verb@{...}@ with
\verb@rep(...)@ and optional occurrence with \verb@[...]@.
Applying this process to the grammar of arithmetic
expressions yields:
\begin{verbatim}
module ExprParser with {
  import Parse;

  def letter   \= =  \= chrWith(c => c.isLetter);
  def digit    \= =  \> chrWith(c => c.isDigit);

  def ident    \> =  \> letter &&& rep(letter ||| digit);
  def number   \> =  \> digit &&& rep(digit);

  def expr:Parser\> =  expr1 &&& rep((chr('+') &&& expr1) ||| (chr('-') &&& expr1));
  def expr1    \> =  expr2 &&& rep((chr('*') &&& expr2) ||| (chr('/') &&& expr2));
  def expr2    \> =  ident ||| number ||| (chr('(') &&& expr &&& chr(')'));
}
\end{verbatim}
It remains to explain how to implement a library with the combinators
described above. We will pack combinators and their underlying
implementation in a module \verb@Parse@.  The first question to decide
is which underlying representation type to use for a parser. We treat
parsers here as functions that take a list of characters as input
parameter and that yield a parse result.
\begin{verbatim}
module Parse with {

  type Result = Option[List[Char]];

  abstract class Parser extends Function1[List[Char],Result] with {
\end{verbatim}
\comment{
The \verb@Option@ type is predefined as follows.
\begin{verbatim}
abstract final class Option[a];
case class None[a] extends Option[a];
case class Some[a](x: a) extends Option[a];
\end{verbatim}
}
A parser returns either the constant \verb@None@, which
signifies that the parser did not recognize a legal input string, or
it returns a value \verb@Some(in1)@ where \verb@in1@ represents that
part of the input list that the parser did not consume.

Parsers are instances of functions from \verb@List[Char]@ to
\verb@Parse.Result@, which also implement the combinators
for sequence and alternative. This is modelled by
defining \verb@Parser@ as a class that extends type
\verb@Function1[List[Char],Result]@ and that defines an \verb@apply@
method, as well as methods \verb@&&&@ and \verb@|||@.
\begin{verbatim}
    abstract def apply(in: List[Char]): Result;
\end{verbatim}
\begin{verbatim}
    def &&& (def p: Parser) = new Parser with {
      def apply(in: List[Char]) = outer.apply(in) match {
        case Some(in1) => p(in1)
        case n => n
      }
    }

    def ||| (def p: Parser) = new Parser with {
      def apply(in: List[Char]) = outer.apply(in) match {
        case None => p(in)
        case s => s
      }
    }
  }
\end{verbatim}
The implementations of the primitive parsers \verb@empty@, \verb@fail@,
\verb@chrWith@ and \verb@chr@ are as follows.
\begin{verbatim}

  def empty = new Parser with { def apply(in: List[Char]) = Some(in) }

  def fail = new Parser with { def apply(in: List[Char]) = None[List[Char]] }

  def chrWith(p: (Char)Boolean) = new Parser with {
    def apply(in: List[Char]) = in match {
      case [] => None[List[Char]]
      case (c :: in1) => if (p(c)) Some(in1) else None[List[Char]]
    }
  }

  def chr(c: Char): Parser = chrWith(d => d == c);
\end{verbatim}
The higher-order parser combinators \verb@opt@ and \verb@rep@ can be
defined in terms of the combinators for sequence and alternative:
\begin{verbatim}
  def opt(p: Parser): Parser = p ||| empty;
  def rep(p: Parser): Parser = opt(rep1(p));
  def rep1(p: Parser): Parser = p &&& rep(p);
} // end Parser
\end{verbatim}
This is all that's needed. Parsers such as the one for arithmetic
expressions given above can now be composed from these building
blocks.  These parsers need not refer to the underlying implementation of
parsers as functions from input lists to parse results.

The presented combinator parsers use backtracking to change from one
alternative to another.  If one restricts the focus to LL(1) grammars,
a non-backtracking implementation of parsers is also possible. This
implementation can then be based on iterators instead of lists.

\section{\label{sec:parsers-results}Parsers that Return Results}

The combinator library of the previous section does not support the
generation of output from parsing. But usually one does not just want
to check whether a given string belongs to the defined language, one
also wants to convert the input string into some internal
representation such as an abstract syntax tree.

In this section, we modify our parser library to build parsers that
produce results. We will make use of the for-comprehensions introduced
in Chapter~\ref{sec:for-notation}.  The basic combinator of sequential
composition, formerly \verb@p &&& q@, now becomes
\begin{verbatim}
for (val x <- p; val y <- q) yield e
\end{verbatim}.
Here, the names \verb@x@ and \verb@y@ are bound to the results of
executing the parsers \verb@p@ and \verb@q@. \verb@e@ is an expression
that uses these results to build the tree returned by the composed
parser.

Before describing the implementation of the new parser combinators, we
explain how the new building blocks are used. Say we want to modify
our arithmetic expression parser so that it returns an abstract syntax
tree of the parsed expression. The class of syntax trees is given by:
\begin{verbatim}
abstract class Tree;
case class Var(n: String) extends Tree;
case class Num(n: Int) extends Tree;
case class Binop(op: Char, l: Tree, r: Tree) extends Tree;
\end{verbatim}
That is, a syntax tree is a named variable, an integer number, or a
binary operation with two operand trees and a character indicating the
operation.

As a first step towards parsers that produce syntax trees, we need to
modify the ``micro-syntax'' parsers \verb@letter@, \verb@digit@,
\verb@ident@ and \verb@number@ so that they return representations of
the parsed input:
\begin{verbatim}
def letter: Parser[Char] = chrWith(c => c.isLetter);
def digit : Parser[Char] = chrWith(c => c.isDigit);

def ident: Parser[String] =
  for (val c <- letter; val cs <- rep(letter ||| digit))
  yield ((c :: cs) foldr "") {c, s => c+ s};

def number: Parser[Int] =
  for (val d <- digit; val ds <- rep(digit))
  yield ((d - '0') :_foldl ds) {x, y => x * 10 + (y - '0')};
\end{verbatim}
The \verb@letter@ and \verb@digit@ parsers simply return the letter
that was parsed. The \verb@ident@ and \verb@number@ parsers return the
string, respectively integer number that was parsed.  In both cases,
sub-parsers are applied in a for-comprehension and their results are
embedded in the result of the calling parser.  The remainder of the
parser for arithmetic expressions follows the same scheme.
\begin{verbatim}
def expr: Parser[Tree] =
  for {
    val e1 <- expr1;
    val es <- rep (
      for {
        val op <- chr('+') ||| chr('-');
	val e <- expr1
      } yield (x => Binop(op, x, e)) : (Tree)Tree
    )
  } yield applyAll(es, e1);
\end{verbatim}
\begin{verbatim}
def expr1: Parser[Tree] =
  for {
    val e1 <- expr2;
    val es <- rep (
      for {
        val op <- chr('*') ||| chr('/');
        val e <- expr2
      } yield (x => Binop(op, x, e)) : (Tree)Tree
    )
  } yield applyAll(es, e1);
\end{verbatim}
\begin{verbatim}
def expr2: Parser[Tree] = {
    \= ( for { val n <- ident } yield Var(n) : Tree )
  |||\> ( for { val n <- number } yield Num(n) : Tree )
  |||\> ( for { val _ <- chr('('); val e <- expr; val _ <- chr(')') } yield e );
}
\end{verbatim}
Note the treatment of the repetitions in \verb@expr@ and
\verb@expr1@. The parser for an expression suffix $op;e$ consisting of an
operator $op$ and an expression $e$ returns a function, which, given a
left operand expression $d$, constructs a \verb@Binop@ node that
represents $d;op;e$. The \verb@rep@ parser combinator forms a list of
all these functions. The final \verb@yield@ part applies all functions
to the first operand in the sequence, which is represented by
\verb@e1@. Here \verb@applyAll@ applies the list of functions passed as its first
argument to its second argument. It is defined as follows.
\begin{verbatim}
def applyAll[a](fs: List[(a)a], e: a): a =
  (e :_foldl fs) { x, f => f(x) }
\end{verbatim}
We now present the parser combinators that support the new
scheme. Parsers that succeed now return a parse result besides the
un-consumed input.
\begin{verbatim}
module Parse with {

  type Result[a] = Option[(a, List[Char])]
\end{verbatim}
Parsers are parameterized with the type of their result. The class
\verb@Parser[a]@ now defines new methods \verb@map@, \verb@flatMap@
and \verb@filter@. The \verb@for@ expressions are mapped by the
compiler to calls of these functions using the scheme described in
Chapter~\ref{sec:for-notation}.

Here is the complete definition of the new \verb@Parser@ class.
\begin{verbatim}
  abstract class Parser[a] extends Function1[List[Char],Result[a]] with {

    def apply(in: List[Char]): Result[a];

    def filter(p: (a)Boolean) = new Parser[a] with {
      def apply(in: List[Char]): Result[a] = outer.apply(in) match {
        case Some((x, in1)) => if (p(x)) Some((x, in1)) else None
        case None => None
      }
    }

    def map[b](f: (a)b) = new Parser[b] with {
      def apply(in: List[Char]): Result[b] = outer.apply(in) match {
	case Some((x, in1)) => Some((f(x), in1))
        case None => None
      }
    }

    def flatMap[b](f: (a)Parser[b]) = new Parser[b] with {
      def apply(in: List[Char]): Result[b] = outer.apply(in) match {
	case Some((x, in1)) => f(x)(in1)
        case None => None
      }
    }

    def ||| (def p: Parser[a]) = new Parser[a] with {
      def apply(in: List[Char]): Result[a] = outer.apply(in) match {
	case None => p(in)
	case s => s
      }
    }

    def &&& [b](def p: Parser[b]): Parser[b] =
      for (val _ <- this; val result <- p) yield result;
  }
\end{verbatim}

The \verb@filter@ method takes as parameter a predicate $p$ which it
applies to the results of the current parser. If the predicate is
false, the parser fails by returning \verb@None@; otherwise it returns
the result of the current parser.  The \verb@map@ method takes as
parameter a function $f$ which it applies to the results of the
current parser. The \verb@flatMap@ tales as parameter a function
\verb@f@ which returns a parser.  It applies \verb@f@ to the result of
the current parser and then continues with the resulting parser.  The
\verb@|||@ method is essentially defined as before.  The
\verb@&&&@ method can now be defined in terms of \verb@for@.

% Here is the code for fail, chrWith and chr
%
%\begin{verbatim}
%  def fail[a] = new Parser[a] with { def apply(in: List[Char]) = None[(a,List[Char])] }
%
%  def chrWith(p: (Char)Boolean) = new Parser[Char] with {
%    def apply(in: List[Char]) = in match {
%      case [] => None[(Char,List[Char])]
%      case (c :: in1) => if (p(c)) Some((c,in1)) else None[(Char,List[Char])]
%    }
%  }
%
%  def chr(c: Char): Parser[Char] = chrWith(d => d == c);
%\end{verbatim}
The primitive parser \verb@succeed@ replaces \verb@empty@. It consumes
no input and returns its parameter as result.
\begin{verbatim}
  def succeed[a](x: a) = new Parser[a] with {
    def apply(in: List[Char]) = Some((x, in))
  }
\end{verbatim}
The \verb@fail@ parser is as before.  The parser combinators
\verb@rep@ and \verb@opt@ now also return results. \verb@rep@ returns
a list which contains as elements the results of each iteration of its
sub-parser. \verb@opt@ returns an
\verb@Option@ type which indicates whether something was recognized by
its sub-parser.
\begin{verbatim}
  def rep[a](p: Parser[a]): Parser[List[a]] =
    rep1(p) ||| succeed([]);

  def rep1[a](p: Parser[a]): Parser[List[a]] =
    for (val x <- p; val xs <- rep(p)) yield x :: xs;

  def opt[a](p: Parser[a]): Parser[Option [a]] =
    { for (val x <- p) yield (Some(x): Option[a]) } ||| succeed((None: Option[a]));
} // end Parse
\end{verbatim}

\chapter{\label{sec:hm}Programming with Patterns: Hindley/Milner Type Inference}

This chapter demonstrates Scala's data types and pattern matching by
developing a type inference system in the Hindley/Milner style. The
source language for the type inferencer is lambda calculus with a let
construct. Abstract syntax trees for the source language are
represented by the following data type of \verb@Terms@.
\begin{verbatim}
abstract class Term;
case class Var(x: String) extends Term;
case class Lam(x: String, e: Term) extends Term;
case class App(f: Term, e: Term) extends Term;
case class Let(x: String, e: Term, f: Term) extends Term;
\end{verbatim}
There are four tree constructors: \verb@Var@ for variables, \verb@Lam@
for function abstractions, \verb@App@ for function applications, and
\verb@Let@ for let expressions. Note that these tree constructors are
defined outside the \verb@Term@ class. It would also be possible
to define further constructors for this type in other parts of the
program.

The next data type describes the form of types that are
computed by the inference system.
\begin{verbatim}
module Types with {
  abstract final class Type;
  case class Tyvar(a: String) extends Type;
  case class Arrow(t1: Type, t2: Type) extends Type;
  case class Tycon(k: String, ts: List[Type]) extends Type;
  private var n: Int = 0;
  def newTyvar: Type = { n = n + 1 ; Tyvar("a" + n) }
}
import Types;
\end{verbatim}
There are three type constructors: \verb@Tyvar@ for type variables,
\verb@Arrow@ for function types and \verb@Tycon@ for type
constructors such as \verb@Boolean@ or \verb@List@. Type constructors
have as component a list of their type parameters. This list is empty
for type constants such as \verb@Boolean@. The data type is packaged
in a module \verb@Types@. Also contained in that module is a function
\verb@newTyvar@ which creates a fresh type variable each time it is
called. The module definition is followed by an import clause
\verb@import Types@, which makes the non-private members of
this module available without qualification in the code that follows.

Note that \verb@Type@ is a \verb@final@ class. This means that no
subclasses or data constructors that extend \verb@Type@ can be formed
except for the three constructors that follow the class.  This makes
\verb@Type@ into a {\em closed} algebraic data type with a fixed
number of alternatives. By contrast, type \verb@Term@ is an {\em open}
algebraic type for which further alternatives can be defined.

The next data type describes type schemes, which consist of a type and
a list of names of type variables which appear universally quantified
in the type scheme. For instance, the type scheme $\forall a\forall
b.a \arrow b$ would be represented in the type checker as:
\begin{verbatim}
TypeScheme(["a", "b"], Arrow(Tyvar("a"), Tyvar("b"))) .
\end{verbatim}
The data type definition of type schemes does not carry an extends
clause; this means that type schemes extend directly class
\verb@Object@.
Even though there is only one possible way to construct a type scheme,
a \verb@case class@ representation was chosen since it offers a convenient
way to decompose a type scheme into its parts using pattern matching.
\begin{verbatim}
case class TypeScheme(ls: List[String], t: Type) with {
  def newInstance: Type = {
    val instSubst =
      ((EmptySubst: Subst) :_foldl ls) { s, a => s.extend(Tyvar(a), newTyvar) }
    instSubst(t)
  }
}
\end{verbatim}
Type scheme objects come with a method \verb@newInstance@, which
returns the type contained in the scheme after all universally type
variables have been renamed to fresh variables.

The next class describes substitutions. A substitution is an
idempotent function from type variables to types. It maps a finite
number of given type variables to given types, and leaves all other
type variables unchanged. The meaning of a substitution is extended
point-wise to a mapping from types to types.

\begin{verbatim}
abstract class Subst extends Function1[Type,Type] with {
  def lookup(x: Tyvar): Type;
  def apply(t: Type): Type = t match {
    case Tyvar(a) => val u = lookup(Tyvar(a)); if (t == u) t else apply(u);
    case Arrow(t1, t2) => Arrow(apply(t1), apply(t2))
    case Tycon(k, ts) => Tycon(k, ts map apply)
  }
  def extend(x: Tyvar, t: Type) = new Subst with {
    def lookup(y: Tyvar): Type = if (x == y) t else outer.lookup(y);
  }
}
case class EmptySubst extends Subst with { def lookup(t: Tyvar): Type = t }
\end{verbatim}
We represent substitutions as functions, of type
\verb@(Type)Type@. To be an instance of this type, a
substitution \verb@s@ has to implement an \verb@apply@ method that takes a
\verb@Type@ as argument and yields another \verb@Type@ as result. A function
application \verb@s(t)@ is then interpreted as \verb@s.apply(t)@.

The \verb@lookup@ method is abstract in class \verb@Subst@.  Concrete
substitutions are defined by the case class \verb@EmptySubst@ and the
method \verb@extend@ in class \verb@Subst@.

The next class gives a naive implementation of sets using lists as the
implementation type. It implements methods \verb@contains@ for
membership tests as well as \verb@union@ and \verb@diff@ for set union
and difference. Alternatively, one could have used a more efficient
implementation of sets in some standard library.
\begin{verbatim}
class ListSet[a](xs: List[a]) with {
  val elems: List[a] = xs;

  def contains(y: a): Boolean = xs match {
    case [] => False
    case x :: xs1 => (x == y) || (xs1 contains y)
  }

  def union(ys: ListSet[a]): ListSet[a] = xs match {
    case [] => ys
    case x :: xs1 =>
      if (ys contains x) ListSet(xs1) union ys
      else ListSet(x :: (ListSet(xs1) union ys).elems)
  }

  def diff(ys: ListSet[a]): ListSet[a] = xs match {
    case [] => ListSet([])
    case x :: xs1 =>
      if (ys contains x) ListSet(xs1) diff ys
      else ListSet(x :: (ListSet(xs1) diff ys).elems)
  }
}
\end{verbatim}

We now present the type checker module. The type checker
computes a type for a given term in a given environment. Environments
associate variable names with type schemes. They are represented by a
type alias \verb@Env@ in module \verb@TypeChecker@:
\begin{verbatim}
module TypeChecker with {

  /** Type environments are lists of bindings that associate a
   * name with a type scheme.
   */
  type Env = List[(String, TypeScheme)];
\end{verbatim}
There is also an exception \verb@TypeError@, which is thrown when type
checking fails. Exceptions are modelled as case classes that inherit
from the predefined \verb@Exception@ class.
\begin{verbatim}
  case class TypeError(msg: String) extends Exception(msg);
\end{verbatim}
The \verb@Exception@ class defines a \verb@throw@ method which causes
the exception to be thrown.

The \verb@TypeChecker@ module contains several utility
functions. Function
\verb@tyvars@ yields the set of free type variables of a type,
of a type scheme, of a list of types, or of an environment. Its
implementation is as four overloaded functions, one for each type of
argument.
\begin{verbatim}
  def tyvars(t: Type): ListSet[String] = t match {
    case Tyvar(a) => new ListSet([a])
    case Arrow(t1, t2) => tyvars(t1) union tyvars(t2)
    case Tycon(k, ts) => tyvars(ts)
  }
  def tyvars(ts: TypeScheme): ListSet[String] = ts match {
    case TypeScheme(as, t) => tyvars(t) diff new ListSet(as)
  }
  def tyvars(ts: List[Type]): ListSet[String] = ts match {
    case [] => new ListSet[String]([])
    case t :: ts1 => tyvars(t) union tyvars(ts1)
  }
  def tyvars(env: Env): ListSet[String] = env match {
    case [] => new ListSet[String]([])
    case (x, t) :: env1 => tyvars(t) union tyvars(env1)
  }
\end{verbatim}
The next utility function, \verb@lookup@, returns the type scheme
associated with a given variable name in the given environment, or
returns \verb@null@ if no binding for the variable exists in the environment.
\begin{verbatim}
  def lookup(env: Env, x: String): TypeScheme = env match {
    case [] => null
    case (y, t) :: env1 => if (x == y) t else lookup(env1, x)
  }
\end{verbatim}
The next utility function, \verb@gen@, returns the type scheme that
results from generalizing a given type in a given environment. This
means that all type variables that occur in the type but not in the
environment are universally quantified.
\begin{verbatim}
  def gen(env: Env, t: Type): TypeScheme =
    TypeScheme((tyvars(t) diff tyvars(env)).elems, t);
\end{verbatim}
The next utility function, \verb@mgu@, computes the most general
unifier of two given types $t$ and $u$ under a pre-existing
substitution $s$.  That is, it returns the most general
substitution $s'$ which extends $s$, and which makes $s'(t)$ and
$s'(u)$ equal types. The function throws a \verb@TypeError@ exception
if no such substitution exists. This can happen because the two types
have different type constructors at corresponding places, or because
a type variable is unified with a type that contains the type variable
itself.
\begin{verbatim}
  def mgu(t: Type, u: Type)(s: Subst): Subst = (s(t), s(u)) match {
    case (Tyvar( a), Tyvar(b)) if a == b =>
      s
    case (Tyvar(a), _) =>
      if (tyvars(u) contains a)
         TypeError("unification failure: occurs check").throw
      else s.extend(Tyvar(a), u)
    case (_, Tyvar(a)) =>
      mgu(u, t)(s)
    case (Arrow(t1, t2), Arrow(u1, u2)) =>
      mgu(t1, u1)(mgu(t2, u2)(s))
    case (Tycon(k1, ts), Tycon(k2, us)) if k1 == k2 =>
      (s :_foldl ((ts zip us) map (case (t,u) => mgu(t,u)))) { s, f => f(s) }
    case _ => TypeError("unification failure").throw
  }
\end{verbatim}
The main task of the type checker is implemented by function
\verb@tp@. This function takes as first parameters an environment $env$, a
term $e$ and a proto-type $t$. As a second parameter it takes a
pre-existing substitution $s$.  The function yields a substitution
$s'$ that extends $s$ and that
turns $s'(env) \ts e: s'(t)$ into a derivable type judgement according
to the derivation rules of the Hindley/Milner type system \cite{hindley-milner}.  A
\verb@TypeError@ exception is thrown if no such substitution exists.
\begin{verbatim}
  def tp(env: Env, e: Term, t: Type)(s: Subst): Subst = e match {
    case Var(x) => {
      val u = lookup(env, x);
      if (u == null) TypeError("undefined: x").throw
      else mgu(u.newInstance, t)(s)
    }
    case Lam(x, e1) => {
      val a = newTyvar, b = newTyvar;
      val s1 = mgu(t, Arrow(a, b))(s);
      val env1 = (x, TypeScheme([], a)) :: env;
      tp(env1, e1, b)(s1)
    }
    case App(e1, e2) => {
      val a = newTyvar;
      val s1 = tp(env, e1, Arrow(a, t))(s);
      tp(env, e2, a)(s1)
    }
    case Let(x, e1, e2) => {
      val a = newTyvar;
      val s1 = tp(env, e1, a)(s);
      tp((x, gen(env, s1(a))) :: env, e2, t)(s1)
    }
  }
\end{verbatim}
The next function, \verb@typeOf@ is a simplified facade for
\verb@tp@. It computes the type of a given term $e$ in a given
environment $env$. It does so by creating a fresh type variable \verb$a$,
computing a typing substitution that makes \verb@env $\ts$ e: a@ into
a derivable type judgement, and finally by returning the result of
applying the substitution to $a$.
\begin{verbatim}
  def typeOf(env: Env, e: Term): Type = {
    val a = newTyvar;
    tp(env, e, a)(EmptySubst)(a)
  }
}
\end{verbatim}
This concludes the presentation of the type inference system.
To apply the system, it is convenient to have a predefined environment
that contains bindings for commonly used constants. The module
\verb@Predefined@ defines an environment \verb@env@ that contains
bindings for booleans, numbers and lists together with some primitive
operations over these types. It also defines a fixed point operator
\verb@fix@, which can be used to represent recursion.
\begin{verbatim}
module Predefined with {
  val booleanType = Tycon("Boolean", []);
  val intType = Tycon("Int", []);
  def listType(t: Type) = Tycon("List", [t]);

  private def gen(t: Type): TypeScheme = TypeChecker.gen([], t);
  private val a = newTyvar;
  val env = [
    ("true", gen(booleanType)),
    ("false", gen(booleanType)),
    ("$\mbox{\prog{if}}$", gen(Arrow(booleanType, Arrow(a, Arrow(a, a))))),
    ("zero", gen(intType)),
    ("succ", gen(Arrow(intType, intType))),
    ("$\mbox{\prog{nil}}$", gen(listType(a))),
    ("cons", gen(Arrow(a, Arrow(listType(a), listType(a))))),
    ("isEmpty", gen(Arrow(listType(a), booleanType))),
    ("head", gen(Arrow(listType(a), a))),
    ("tail", gen(Arrow(listType(a), listType(a)))),
    ("fix", gen(Arrow(Arrow(a, a), a)))
  ];
}
\end{verbatim}
Here's an example how the type inferencer is used.
Let's define a function \verb@showType@ which returns the type of
a given term computed in the predefined environment
\verb@Predefined.env@:
\begin{verbatim}
> def showType(e: Term) = TypeChecker.typeOf(Predefined.env, e);
\end{verbatim}
Then the application
\begin{verbatim}
> showType(Lam("x", App(App(Var("cons"), Var("x")), Var("$\mbox{\prog{nil}}$"))));
\end{verbatim}
would give the response
\begin{verbatim}
> TypeScheme([a0], Arrow(Tyvar(a0), Tycon("List", [Tyvar(a0)])));
\end{verbatim}

\paragraph{Exercise:}
Add \verb@toString@ methods to the data constructors of class
\verb@Type@ and \verb@TypeScheme@ which represent types in a more
natural way.

\chapter{Abstractions for Concurrency}\label{sec:ex-concurrency}

This section reviews common concurrent programming patterns and shows
how they can be implemented in Scala.

\section{Signals and Monitors}

\example Th {\em monitor} provides the basic means for mutual exclusion
of processes in Scala. It is defined as follows.
\begin{verbatim}
class Monitor with {
  def synchronized [a] (def e: a): a;
}
\end{verbatim}
The \verb@synchronized@ method in class \verb@Monitor@ executes its
argument computation \verb@e@ in mutual exclusive mode -- at any one
time, only one thread can execute a \verb@synchronized@ argument of a
given monitor.

Threads can suspend inside a monitor by waiting on a signal.  The
\verb@Signal@ class offers two methods \verb@send@ and
\verb@wait@.  Threads that call the \verb@wait@ method wait until a
\verb@send@ method of the same signal is called subsequently by some
other thread. Calls to \verb@send@ with no threads waiting for the
signal are ignored. Here is the specification of the \verb@Signal@
class.
\begin{verbatim}
class Signal with {
  def wait: Unit;
  def wait(msec: Long): Unit;
  def notify: Unit;
  def notifyAll: Unit;
}
\end{verbatim}
A signal also implements a timed form of \verb@wait@, which blocks
only as long as no signal was recieved or the specified amount of time
(given in milliseconds) has elaosed. Furthermore, there is a
\verb@notifyAll@ method which unblocks all threads which wait for the
signal. \verb@Signal@ and \verb@Monitor@ are primitive classes in
Scala which are implemented in terms of the underlying runtime system.

As an example of how monitors and signals are used, here is is an
implementation of a bounded buffer class.
\begin{verbatim}
class BoundedBuffer[a](N: int) extends Monitor with {
  var in = 0, out = 0, n = 0;
  val elems = new Array[a](N);
  val nonEmpty = new Signal;
  val nonFull = new Signal;
\end{verbatim}
\begin{verbatim}
  def put(x: a) = synchronized {
    if (n == N) nonFull.wait;
    elems(in) = x ; in = (in + 1) % N ; n = n + 1;
    if (n == 1) nonEmpty.send;
  }
\end{verbatim}
\begin{verbatim}
  def get: a = synchronized {
    if (n == 0) nonEmpty.wait
    val x = elems(out) ; out = (out + 1) % N ; n = n - 1;
    if (n == N - 1) nonFull.send;
    x
  }
}
\end{verbatim}
And here is a program using a bounded buffer to communicate between a
producer and a consumer process.
\begin{verbatim}
val buf = new BoundedBuffer[String](10)
fork { while (True) { val s = produceString ; buf.put(s) } }
fork { while (True) { val s = buf.get ; consumeString(s) } }
\end{verbatim}
The \verb@fork@ method spawns a new thread which executes the
expression given in the parameter. It can be defined as follows.
\begin{verbatim}
def fork(def e: Unit) = {
  val p = new Thread with { def run = e; }
  p.run
}
\end{verbatim}

\comment{
\section{Logic Variable}

A logic variable (or lvar for short) offers operations \verb@:=@
and \verb@value@ to define the variable and to retrieve its value.
Variables can be \verb@define@d only once. A call to \verb@value@
blocks until the variable has been defined.

Logic variables can be implemented as follows.

\begin{verbatim}
class LVar[a] extends Monitor with {
  private val defined = new Signal
  private var isDefined: Boolean = False
  private var v: a
  def value = synchronized {
    if (!isDefined) defined.wait
    v
  }
  def :=(x: a) = synchronized {
    v = x ; isDefined = True ; defined.send
  }
}
\end{verbatim}
}

\section{SyncVars}

A synchronized variable (or syncvar for short) offers \verb@get@ and
\verb@put@ operations to read and set the variable. \verb@get@ operations
block until the variable has been defined. An \verb@unset@ operation
resets the variable to undefined state.

Synchronized variables can be implemented as follows.
\begin{verbatim}
class SyncVar[a] extends Monitor with {
  private val defined = new Signal;
  private var isDefined: Boolean = False;
  private var value: a;
  def get = synchronized {
    if (!isDefined) defined.wait;
    value
  }
  def set(x: a) = synchronized {
    value = x ; isDefined = True ; defined.send;
  }
  def isSet: Boolean =
    isDefined;
  def unset = synchronized {
    isDefined = False;
  }
}
\end{verbatim}

\section{Futures}
\label{sec:futures}

A {\em future} is a value which is computed in parallel to some other
client thread, to be used by the client thread at some future time.
Futures are used in order to make good use of parallel processing
resources.  A typical usage is:

\begin{verbatim}
val x = future(someLengthyComputation);
anotherLengthyComputation;
val y = f(x()) + g(x());
\end{verbatim}

Futures can be implemented in Scala as follows.

\begin{verbatim}
def future[a](def p: a): (Unit)a = {
  val result = new SyncVar[a];
  fork { result.set(p) }
  (=> result.get)
}
\end{verbatim}

The \verb@future@ method gets as parameter a computation \verb@p@ to
be performed. The type of the computation is arbitrary; it is
represented by \verb@future@'s type parameter \verb@a@.  The
\verb@future@ method defines a guard \verb@result@, which takes a
parameter representing the result of the computation. It then forks
off a new thread that computes the result and invokes the
\verb@result@ guard when it is finished. In parallel to this thread,
the function returns an anonymous function of type \verb@a@.
When called, this functions waits on the result guard to be
invoked, and, once this happens returns the result argument.
At the same time, the function reinvokes the \verb@result@ guard with
the same argument, so that future invocations of the function can
return the result immediately.

\section{Parallel Computations}

The next example presents a function \verb@par@ which takes a pair of
computations as parameters and which returns the results of the computations
in another pair. The two computations are performed in parallel.

\begin{verbatim}
def par[a, b](def xp: a, def yp: b): (a, b) = {
  val y = new SyncVar[a];
  fork { y.set(yp) }
  (xp, y)
}
\end{verbatim}

The next example presents a function \verb@replicate@ which performs a
number of replicates of a computation in parallel. Each
replication instance is passed an integer number which identifies it.

\begin{verbatim}
def replicate(start: Int, end: Int)(def p: (Int)Unit): Unit = {
  if (start == end) {
  } else if (start + 1 == end) {
    p(start)
  } else {
    val mid = (start + end) / 2;
    par ( replicate(start, mid)(p), replicate(mid, end)(p) )
  }
}
\end{verbatim}

The next example shows how to use \verb@replicate@ to perform parallel
computations on all elements of an array.

\begin{verbatim}
def parMap[a,b](f: (a)b, xs: Array[a]): Array[b] = {
  val results = new Array[b](xs.length);
  replicate(0, xs.length) { i => results(i) = f(xs(i)) }
  results
}
\end{verbatim}

\section{Semaphores}

A common mechanism for process synchronization is a {\em lock} (or:
{\em semaphore}). A lock offers two atomic actions: \prog{acquire} and
\prog{release}. Here's the implementation of a lock in Scala:

\begin{verbatim}
class Lock extends Monitor with Signal with {
  var available = True;
  def acquire = {
    if (!available) wait;
    available = False
  }
  def release = {
    available = True;
    notify
  }
}
\end{verbatim}

\section{Readers/Writers}

A more complex form of synchronization distinguishes between {\em
readers} which access a common resource without modifying it and {\em
writers} which can both access and modify it. To synchronize readers
and writers we need to implement operations \prog{startRead}, \prog{startWrite},
\prog{endRead}, \prog{endWrite}, such that:
\begin{itemize}
\item there can be multiple concurrent readers,
\item there can only be one writer at one time,
\item pending write requests have priority over pending read requests,
but don't preempt ongoing read operations.
\end{itemize}
The following implementation of a readers/writers lock is based on the
{\em message space} concept (see Section~\ref{sec:messagespace}).

\begin{verbatim}
class ReadersWriters with {
  val m = new MessageSpace;
  private case class Writers(n: Int), Readers(n: Int);
  Writers(0); Readers(0);
  def startRead = m receive {
    case Writers(n) if n == 0 => m receive {
      case Readers(n) => Writers(0) ; Readers(n+1);
    }
  }
  def startWrite = m receive {
    case Writers(n) =>
      Writers(n+1);
      m receive { case Readers(n) if n == 0 => }
  }
\end{verbatim}
\begin{verbatim}
  def endRead = receive {
    case Readers(n) => Readers(n-1)
  }
  def endWrite = receive {
    case Writers(n) => Writers(n-1) ; if (n == 0) Readers(0)
  }
}
\end{verbatim}

\section{Asynchronous Channels}

A fundamental way of interprocess comunication is the asynchronous
channel. Its implementation makes use the following class for linked
lists:
\begin{verbatim}
class LinkedList[a](x: a) with {
  val elem: a = x;
  var next: LinkedList[a] = null;
}
\end{verbatim}
To facilite insertion and deletion of elements into linked lists,
every reference into a linked list points to the node which precedes
the node which conceptually forms the top of the list.
Empty linked lists start with a dummy node, whose successor is \verb@null@.

The channel class uses a linked list to store data that has been sent
but not read yet. In the opposite direction, a signal \verb@moreData@ is
used to wake up reader threads that wait for data.
\begin{verbatim}
class Channel[a] with {
  private val written = new LinkedList[a](null);
  private var lastWritten = written;
  private val moreData = new Signal;

  def write(x: a) = {
    lastWritten.next = new LinkedList(x);
    lastWritten = lastWritten.next;
    moreData.notify;
  }

  def read: a = {
    if (written.next == null) moreData.wait;
    written = written.next;
    written.elem;
  }
}
\end{verbatim}

\section{Synchronous Channels}

Here's an implementation of synchronous channels, where the sender of
a message blocks until that message has been received. Synchronous
channels only need a single variable to store messages in transit, but
three signals are used to coordinate reader and writer processes.
\begin{verbatim}
class SyncChannel[a] with {
  val data = new SyncVar[a];

  def write(x: a): Unit = synchronized {
    val empty = new Signal, full = new Signal, idle = new Signal;
    if (data.isSet) idle.wait;
    data.put(x);
    full.send;
    empty.wait;
    data.unset;
    idle.send;
  }

  def read: a = synchronized {
    if (!(data.isSet)) full.wait;
    x = data.get;
    empty.send;
    x
  }
}
\end{verbatim}

\section{Workers}

Here's an implementation of a {\em compute server} in Scala. The
server implements a \verb@future@ method which evaluates a given
expression in parallel with its caller. Unlike the implementation in
Section~\ref{sec:futures} the server computes futures only with a
predefined number of threads. A possible implementation of the server
could run each thread on a separate processor, and could hence avoid
the overhead inherent in context-switching several threads on a single
processor.

\begin{verbatim}
class ComputeServer(n: Int) {
  private abstract class Job with {
    abstract type t;
    def task: t;
    def return(x: t): Unit;
  }

  private val openJobs = new Channel[Job]

  private def processor: Unit = {
    while (True) {
      val job = openJobs.read;
      job.return(job.task)
    }
  }
\end{verbatim}
\begin{verbatim}
  def future[a](def p: a): ()a = {
    val reply = new SyncVar[a];
    openJobs.write(
      new Job with {
	type t = a;
	def task = p;
	def return(x: a) = reply.set(x);
      }
    )
    (=> reply.get)
  }

  replicate(n){processor};
}
\end{verbatim}

Expressions to be computed (i.e. arguments
to calls of \verb@future@) are written to the \verb@openJobs@
channel. A {\em job} is an object with
\begin{itemize}
\item
An abstract type \verb@t@ which describes the result of the compute
job.
\item
A parameterless \verb@task@ method of type \verb@t@ which denotes
the expression to be computed.
\item
A \verb@return@ method which consumes the result once it is
computed.
\end{itemize}
The compute server creates $n$ \verb@processor@ processes as part of
its initialization.  Every such process repeatedly consumes an open
job, evaluates the job's \verb@task@ method and passes the result on
to the job's
\verb@return@ method. The polymorphic \verb@future@ method creates
a new job where the \verb@return@ method is implemented by a guard
named \verb@reply@ and inserts this job into the set of open jobs by
calling the \verb@isOpen@ guard. It then waits until the corresponding
\verb@reply@ guard is called.

The example demonstrates the use of abstract types. The abstract type
\verb@t@ keeps track of the result type of a job, which can vary
between different jobs. Without abstract types it would be impossible
to implement the same class to the user in a statically type-safe
way, without relying on dynamic type tests and type casts.

\section{Message Spaces}
\label{sec:messagespace}

Message spaces are high-level, flexible constructs for process
synchronization and communication. A {\em message} in this context is
an arbitrary object.  There is a special message \verb@TIMEOUT@ which
is used to signal a time-out.
\begin{verbatim}
case class TIMEOUT;
\end{verbatim}
Message spaces implement the following signature.
\begin{verbatim}
class MessageSpace with {
  def send(msg: Any): Unit;
  def receive[a](f: PartialFunction[Any, a]): a;
  def receiveWithin[a](msec: Long)(f: PartialFunction[Any, a]): a;
}
\end{verbatim}
The state of a message space consists of a multi-set of messages.
Messages are added to the space using the \verb@send@ method. Messages
are removed using the \verb@receive@ method, which is passed a message
processor \verb@f@ as argument, which is a partial function from
messages to some arbitrary result type. Typically, this function is
implemented as a pattern matching expression. The \verb@receive@
method blocks until there is a message in the space for which its
message processor is defined.  The matching message is then removed
from the space and the blocked thread is restarted by applying the
message processor to the message. Both sent messages and receivers are
ordered in time. A receiver $r$ is applied to a matching message $m$
only if there is no other (message, receiver) pair which precedes $(m,
r)$ in the partial ordering on pairs that orders each component in
time.

As a simple example of how message spaces are used, consider a
one-place buffer:
\begin{verbatim}
class OnePlaceBuffer with {
  private val m = new MessageSpace;        \=// An internal message space
  private case class Empty, Full(x: Int);    \>// Types of messages we deal with

  m send Empty;                           \>// Initialization

  def write(x: Int): Unit =
    m receive { case Empty => m send Full(x) }

  def read: Int =
    m receive { case Full(x) => m send Empty ; x }
}
\end{verbatim}
Here's how the message space class can be implemented:
\begin{verbatim}
class MessageSpace with {

  private abstract class Receiver extends Signal with {
    def isDefined(msg: Any): Boolean;
    var msg = null;
  }
\end{verbatim}
We define an internal class for receivers with a test method
\verb@isDefined@, which indicates whether the receiver is
defined for a given message.  The receiver inherits from class
\verb@Signal@ a \verb@notify@ method which is used to wake up a
receiver thread. When the receiver thread is woken up, the message it
needs to be applied to is stored in the \verb@msg@ variable of
\verb@Receiver@.
\begin{verbatim}
  private val sent = new LinkedList[Any](null) ;
  private var lastSent = sent;
  private var receivers = new LinkedList[Receiver](null);
  private var lastReceiver = receivers;
\end{verbatim}
The message space class maintains two linked lists,
one for sent but unconsumed messages, the other for waiting receivers.
\begin{verbatim}
  def send(msg: Any): Unit = synchronized {
    var r = receivers, r1 = r.next;
    while (r1 != null && !r1.elem.isDefined(msg)) {
      r = r1; r1 = r1.next;
    }
    if (r1 != null) {
      r.next = r1.next; r1.elem.msg = msg; r1.elem.notify;
    } else {
      l = new LinkedList(msg); lastSent.next = l; lastSent = l;
    }
  }
\end{verbatim}
The \verb@send@ method first checks whether a waiting receiver is

applicable to the sent message. If yes, the receiver is notified.
Otherwise, the message is appended to the linked list of sent messages.
\begin{verbatim}
  def receive[a](f: PartialFunction[Any, a]): a = {
    val msg: Any = synchronized {
      var s = sent, s1 = s.next;
      while (s1 != null && !f.isDefined(s1.elem)) {
	s = s1; s1 = s1.next
      }
      if (s1 != null) {
        s.next = s1.next; s1.elem
      } else {
	val r = new LinkedList(
          new Receiver with {
            def isDefined(msg: Any) = f.isDefined(msg);
          });
	lastReceiver.next = r; lastReceiver = r;
	r.elem.wait;
	r.elem.msg
      }
    }
    f(msg)
  }
\end{verbatim}
The \verb@receive@ method first checks whether the message processor function
\verb@f@ can be applied to a message that has already been sent but that
was not yet consumed. If yes, the thread continues immediately by
applying \verb@f@ to the message. Otherwise, a new receiver is created
and linked into the \verb@receivers@ list, and the thread waits for a
notification on this receiver. Once the thrad is woken up again, it
continues by applying \verb@f@ to the message that was stored in te receiver.

The message space class also offers a method \verb@receiveWithin@
which blocks for only a specified maximal amount of time.  If no
message is received within the specified time interval (given in
milliseconds), the message processor argument $f$ will be unblocked
with the special \verb@TIMEOUT@ message.  The implementation of
\verb@receiveWithin@ is quite similar to \verb@receive@:
\begin{verbatim}
  def receiveWithin[a](msec: Long)(f: PartialFunction[Any, a]): a = {
    val msg: Any = synchronized {
      var s = sent, s1 = s.next;
      while (s1 != null && !f.isDefined(s1.elem)) {
	s = s1; s1 = s1.next ;
      }
      if (s1 != null) {
        s.next = s1.next; s1.elem
      } else {
	val r = new LinkedList(
          new Receiver with {
            def isDefined(msg: Any) = f.isDefined(msg);
          }
        )
	lastReceiver.next = r; lastReceiver = r;
	r.elem.wait(msec);
        if (r.elem.msg == null) r.elem.msg = TIMEOUT;
	r.elem.msg
      }
    }
    f(msg)
  }
} // end MessageSpace
\end{verbatim}
The only differences are the timed call to \verb@wait@, and the
statement following it.

\section{Actors}
\label{sec:actors}

Chapter~\ref{sec:ex-auction} sketched as a program example the
implementation of an electronic auction service. This service was
based on high-level actor processes, that work by inspecting messages
in their mailbox using pattern matching. An actor is simply a thread
whose communication primitives are those of a message space.
Actors are therefore defined by a mixin composition of threads and message spaces.
\begin{verbatim}
abstract class Actor extends Thread with MessageSpace;
\end{verbatim}

\comment{
As an extended example of an application that uses actors, we come
back to the auction server example of Section~\ref{sec:ex-auction}.
The following code implements:

\begin{figure}[h]
\begin{verbatim}
class AuctionMessage;
case class
  \=Offer(bid: Int, client: Process),                             \=// make a bid
     \>Inquire(client: Process) extends AuctionMessage  \>// inquire status

class AuctionReply;
case class
  \=Status(asked; Int, expiration: Date),   \>// asked sum, expiration date
     \>BestOffer,                                       \>// yours is the best offer
     \>BeatenOffer(maxBid: Int),                        \>// offer beaten by maxBid
     \>AuctionConcluded(seller: Process, client: Process), \>// auction concluded
     \>AuctionFailed                                     \>// failed with no bids
     \>AuctionOver extends AuctionReply                  \>// bidding is closed
\end{verbatim}
\end{figure}

\begin{verbatim}
class Auction(seller: Process, minBid: Int, closing: Date)
 extends Process with {

  val timeToShutdown = 36000000 // msec
  val delta = 10                // bid increment
\end{verbatim}
\begin{verbatim}
  def run = {
    var askedBid = minBid
    var maxBidder: Process = null
    while (True) {
      receiveWithin ((closing - Date.currentDate).msec) {
	case Offer(bid, client) => {
	  if (bid >= askedBid) {
            if (maxBidder != null && maxBidder != client) {
              maxBidder send BeatenOffer(bid)
            }
            maxBidder = client
            askedBid = bid + delta
            client send BestOffer
          } else {
            client send BeatenOffer(maxBid)
          }
        }
\end{verbatim}
\begin{verbatim}
	case Inquire(client) => {
	  client send Status(askedBid, closing)
        }
\end{verbatim}
\begin{verbatim}
	case TIMEOUT => {
	  if (maxBidder != null) {
	    val reply = AuctionConcluded(seller, maxBidder)
	    maxBidder send reply
	    seller send reply
	  } else {
	    seller send AuctionFailed
          }
          receiveWithin (timeToShutdown) {
            case Offer(_, client) => client send AuctionOver ; discardAndContinue
            case _ => discardAndContinue
            case TIMEOUT => stop
          }
        }
\end{verbatim}
\begin{verbatim}
        case _ => discardAndContinue
      }
    }
  }
\end{verbatim}
\begin{verbatim}
  def houseKeeping: Int = {
    val Limit = 100
    var nWaiting: Int = 0
    receiveWithin(0) {
      case _ =>
        nWaiting = nWaiting + 1
        if (nWaiting > Limit) {
	  receiveWithin(0) {
            case Offer(_, _) => continue
            case TIMEOUT =>
            case _ => discardAndContinue
          }
        } else continue
      case TIMEOUT =>
    }
  }
}
\end{verbatim}
\begin{verbatim}
class Bidder (auction: Process, minBid: Int, maxBid: Int)
 extends Process with {
  val MaxTries = 3
  val Unknown = -1

  var nextBid = Unknown
\end{verbatim}
\begin{verbatim}
  def getAuctionStatus = {
    var nTries = 0
    while (nextBid == Unknown && nTries < MaxTries) {
      auction send Inquiry(this)
      nTries = nTries + 1
      receiveWithin(waitTime) {
        case Status(bid, _) => bid match {
          case None => nextBid = minBid
          case Some(curBid) => nextBid = curBid + Delta
        }
        case TIMEOUT =>
        case _ => continue
      }
    }
    status
  }
\end{verbatim}
\begin{verbatim}
  def bid: Unit = {
    if (nextBid < maxBid) {
      auction send Offer(nextBid, this)
      receive {
        case BestOffer =>
          receive {
            case BeatenOffer(bestBid) =>
              nextBid = bestBid + Delta
              bid
            case AuctionConcluded(seller, client) =>
     	      transferPayment(seller, nextBid)
            case _ => continue
	  }

        case BeatenOffer(bestBid) =>
          nextBid = nextBid + Delta
          bid

        case AuctionOver =>

        case _ => continue
      }
    }
  }
\end{verbatim}
\begin{verbatim}
  def run = {
    getAuctionStatus
    if (nextBid != Unknown) bid
  }

  def transferPayment(seller: Process, amount: Int)
}
\end{verbatim}
}
%\todo{We also need some XML examples.}
\end{document}



  case ([], _) => ys
  case (_, []) => xs
  case (x :: xs1, y :: ys1) =>
    if (x < y) x :: merge(xs1, ys) else y :: merge(xs, ys1)
}

def split (xs: List[a]): (List[a], List[a]) = xs match {
  case [] => ([], [])
  case [x] => (x, [])
  case y :: z :: xs1 => val (ys, zs) = split(xs1) ; (y :: ys, z :: zs)
}

def sort(xs: List[a]): List[a] = {
  val (ys, zs) = split(xs)
  merge(sort(ys), sort(zs))
}


def sort(a:Array[String]): Array[String] = {
  val pivot = a(a.length / 2)
  sort(a.filter(x => x < pivot)) ++
       a.filter(x => x == pivot) ++
  sort(a.filter(x => x > pivot))
}

def sort(a:Array[String]): Array[String] = {

  def swap (i: Int, j: Int): unit = {
    val t = a(i) ; a(i) = a(j) ; a(j) = t
  }

  def sort1(l: int, r: int): unit = {
    val pivot = a((l + r) / 2)
    var i = l, j = r
    while (i <= r) {
      while (i < r && a(i) < pivot) { i = i + 1 }
      while (j > l && a(j) > pivot) { j = j - 1 }
      if (i <= j) {
        swap(i, j)
        i = i + 1
        j = j - 1
      }
    }
    if (l < j) sort1(l, j)
    if (j < r) sort1(i, r)
  }

  sort1(0, a.length - 1)
}

class Array[a] with {

  def copy(to: Array[a], src: int, dst: int, len: int): unit
  val length: int
  val apply(i: int): a
  val update(i: int, x: a): unit

  def filter (p: a => Boolean): Array[a] = {
    val temp = new Array[a](a.length)
    var i = 0, j = 0
    for (i < a.length, i = i + 1) {
      val x = a(i)
      if (p(x)) { temp(j) = x; j = j + 1 }
    }
    val res = new Array[a](j)
    temp.copy(res, 0, 0, j)
  }

  def ++ (that: Array[a]): Array[a] = {
    val a = new Array[a](this.length + that.length)
    this.copy(a, 0, 0, this.length)
    that.copy(a, 0, this.length, that.length)
  }

static

  def concat [a] (as: List[Array[a]]) = {
    val l = (as map (a => a.length)).sum
    val dst = new Array[a](l)
    var j = 0
    as forall {a => { a.copy(dst, j, a.length) ; j = j + a.length }}
    dst
  }

}

module ABT extends AlgBinTree[kt, vt]
ABT.Map
